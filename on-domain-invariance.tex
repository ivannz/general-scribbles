% https://terrytao.wordpress.com/2011/06/13/brouwers-fixed-point-and-invariance-of-domain-theorems-and-hilberts-fifth-problem/#more-4937

Consider a subset $U$ which is open in $\Omega$ and an injective continuous map
$f\colon (U, \Tcal_U) \to (S, \Tcal_S)$. Then $V = f(U)$ is open in $S$ and $f$
is a homeomorphism $U \to V$. Here $\Tcal_U$ is subspace topologies, but $\Tcal_S$
is an unfortunate notation.

Since $U$ is open in $(\Omega, \Tcal)$ the subspace topology $\Tcal_U$ is coarser
than $\Tcal$: $\Tcal_U \subseteq \Tcal$.

The map $f^V \colon U \to V$ defined as $f^V(\omega) = f(\omega)$ for all $\omega
\in U$ is $(U, \Tcal_U)$-$(V, \Tcal_V)$ continuous. Indeed, if $W$ is open in
$(V, \Tcal_V)$, then there is $O$ open in $S$ such that $W = V \cap O$. Hence 
\begin{align*}
  \Bigl(f^V\Bigr)^{-1}(W)
    &= \Bigl(f^V\Bigr)^{-1}\bigl(V \cap W\bigr)
    = f^{-1}\bigl(V\cap W\bigr)
    = f^{-1}\bigl(V\cap O\bigr)
    \\
    &= f^{-1}\bigl(f(U)\bigr) \cap f^{-1}(O)
    = U \cap f^{-1}(O)
    \in \Tcal_U
  \,.
\end{align*}
The map $f^V$ is also a bijection, since for any $x\in V$ by definition, there is
$\omega \in U$ such that $f^V(\omega) = f(\omega) = x$. Therefore without losing
generality we may consider a continuous bijection $f\colon (U, \Tcal_U) \to (S, \Tcal_S)$.

So, consider a continuous bijection $f\colon (U, \Tcal_U) \to (V, \Tcal_V)$, where
$U, V$ are open in $\real^n$ and $\Tcal_U$ and $\Tcal_V$ are subspace topologies on
$U$ and $V$, repectively.

The space $\real^n$ is homeomorphic to itself, so there must be a continuous bijection
$\phi\colon \real^n \to \real^n$ with continuous inverse.

% $\real^n$ is a normed space, Hausdorf, separable, countable topological base, closed
% subsets of a compact set are compact, has vector space structure over the field of
% $\real$, 

% for any $a\in \real^n$ let $\tau_a(v) = v + a$ be the translation map. $\tau_a$ is
% an $\real^n\leftrightarrow \real^n$ homeomorphism for any $a$ with $\tau_a^{-1} = \tau_{-a}$.
% Similar properties hold for the scaling map $v\mapsto \sigma_\alpha(x) = \tfrac1\alpha x$
% for any $\alpha \neq 0$

% \tilde{f} = f \circ \tau_a^{-1} is a continuous bijection $\tau_a^{-1}(U)\to V$. However,
% $0 \in \tau_a^{-1}(U)$

% consider a homeomorphism $\psi\colon (-1, +1) \to \real$. Then $\psi_+\Big\vert_{[0, +1)}$
% is a homeomorphism between $[0, +1)$ and $\real_+$ (seen as topological subspaces).

% Do we need $\psi_+^{-1}$ to be linear or $o(x)$ for $x$ sufficiently close to $0$?

% The map $\phi(x) = \psi_+^{-1}\bigl(\|x\|\bigr) \tfrac{x}{\|x\|}$ is a product of 
% continuous maps $\alpha \mapsto \tfrac{\psi_+^{-1}(\alpha)}{\alpha}$ and the identity,
% and its inverse is continuous: $y = \phi(x)$ $\Rightarrow$ $\|y\| = \psi_+^{-1}(\|x\|)$,
% whence $\|x\| = \psi_+(\|y\|)$ and $\phi^{-1}(y) = \tfrac{\psi_+(\|y\|)}{\|y\|} y$.

% Thus $\phi$ is a homeomorphism between $\{x\in\real^n\colon \|x\| < 1\}$ and $\real^n$.

% a continuous bijection between compact Hausdorff spaces and is thus a homeomorphism.
