\section{On Douglas-Rachford splittting} % (fold)
\label{sec:on_douglas_rachford_splittting}

This a short recap of a post at regularize.\footnote
\footnotetext{\url{https://regularize.wordpress.com/2017/02/24/the-origin-of-the-douglas-rachford-iteration/}}

Consider a Hilbert space $(\mathcal{H}, \langle \cdot, \cdot\rangle)$. A
point-set map $T\colon \mathcal{H} \to \mathcal{P}(\mathcal{H})$ is a monotone
operator if for any $x, y \in \mathcal{H}$ and any $u\in Tx$ and $v\in Ty$
we have
$
\langle
    x-y,
    u-v
\rangle \geq 0
$.

A problem of the form: find $x\in \mathcal{H}$ such that $0 \in Tx$, --
is called a \textbf{monotone inclusion}. It is typically solved via splitting.

We first split an operator, then find the resolvents of the components, and
finally formulate a fixed point problem, that define an iterative algorithm.

\subsection{Splitting} % (fold)
\label{sub:splitting}

% subsection splitting (end)

% section on_douglas_rachford_splittting (end)
