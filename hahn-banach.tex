\documentclass[a4paper]{article}

\usepackage[T1]{fontenc}
\usepackage[utf8]{inputenc}

\usepackage{amsmath}
\usepackage{amssymb}
\usepackage{xcolor}
\usepackage{amsthm}
\usepackage[mathcal]{euscript}

\usepackage{url}

\newcommand{\Hcal}{\mathcal{H}}
\newcommand{\real}{\mathbb{R}}
\newcommand{\interior}[1]{%
  {\kern0pt#1}^{\mathrm{o}}%
}
\newcommand{\Tcal}{\mathcal{T}}
\newcommand{\Lcal}{\mathcal{L}}

\title{Notes on Minkowski Functional and Hahn-Banach}
\author{Nazarov Ivan}

\date{\today}

\begin{document}
\maketitle

Consider a Banach space $(\Hcal, \|\cdot\|)$ with the norm $\Tcal^{\|\cdot\|}_\Hcal$
on it $\Hcal$. In the following $[C]$ is the closure of a set $C$ in the topology
of $\Hcal$: $[C]$ the smallest closed set covering $C$, $[C] = \bigl[[C]\bigr]$,
and monotone. In contrast, $\interior{C}$ --- the topological interior of $C$ in
$\Hcal$, --- is the largest open set contained within $C$.

For a nonempty $C \subseteq \Hcal$ the {\it Minkowski Functional} $p\colon \Hcal
\to [0, +\infty]$ is
\begin{equation*}
  p(x)
    = \inf \bigl\{ \lambda > 0\,:\, x \in \lambda C\bigr\}
    \,,
\end{equation*}
where $\lambda C = \{\lambda z \,:\, z\in C\}$ is the $\lambda$-inflation of $C$,
$\lambda \in\real$. This functional has many useful properties when $C$ is convex
and the interior of $C$ contains $0$.

\paragraph{Hahn-Banach point-set separation} % (fold)
\label{par:hahn_banach_point_set_separation}

Let $C$ be a nonempty, closed, convex set with $0\in \interior{C}$. Then, for any
$x_0\notin C$ there is $l_{x_0} \in \Hcal^*$, a continuous linear functional on
$\Hcal$, such that $l_{x_0}(x) \leq 1 < l_{x_0}(x_0)$ for $x\in C$.

Let $p$ be the Minkowski functional of the set $C$, and consider a linear subspace
$\Lcal_{x_0} = \{t x_0\colon t\in \real\}$. $\Lcal_{x_0}$ is nontrivial, since $0\in C$
and $x_0\notin C$. Observe that for a given $x\in \Lcal_{x_0}$ there is a unique
$t_x \in \real$ such that $x = t_x x_0$. Indeed, $t_1\neq t_2$ with $t_1 x_0 = t_2 x_0$
would imply $x_0 = 0$.

The map $\tau\colon \Lcal_{x_0} \to \real$ defined by $\tau(x) = t_x$ is one-to-one
and linear. Indeed, if $x, y\in \Lcal_{x_0}$, then $\tau(x) = \tau(y)$ implies that
$x = \tau(x) x_0 = \tau(y) x_0 = y$, and for any $\alpha\in \real$
\begin{equation}
  \bigl( \tau(x) x_0 \bigr) + \bigl( \tau(y) x_0 \bigr) \alpha
    % structure of $x$ and $y$
    = x + y \alpha
    % linearity of $\Lcal_{x_0}$
    = \tau(x + y \alpha) x_0
    % uniquenes of x\mapsto t_x
  \,.
\end{equation}

Since $x_0 \notin C$ and $C$ is closed, we must have $x_0\notin \{p\leq 1\}$, whence
$p(x_0) > 1$. Let $f(x) = \tau(x) p(x_0)$ for any $x \in \Lcal_{x_0}$. On $\Lcal_{x_0}$
we have $f\leq p$ since $\tau(x) p(x_0) \leq p\bigl(\tau(x) x_0\bigr) = p(x)$ by
sublinearity of $p$, which follows from the {\bf nice} properties of $C$. At the same
time $f$ inherits linearity from $\tau$.

By the {\it Hahn-Banach Extension Theorem} there must be linear extension $l_{x_0}$
of $f$ from $\Lcal_{x_0}$ onto $\Hcal$ with $l_{x_0}\big\vert_{\Lcal_{x_0}} = f$ and
$l_{x_0} \leq p$ everywhere on $\Hcal$. This extension is a bounded linear functional
by an earlier result, and satisfies $l_{x_0}(x) \leq p(x) \leq 1$ for any $x\in C$
and $1 < p(x_0) = \tau(x_0) p(x_0) = f(x_0) = l_{x_0}(x_0)$.

% paragraph hahn_banach_point_set_separation (end)

\end{document}
