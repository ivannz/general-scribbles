\documentclass[a4paper]{article}

\usepackage[T1]{fontenc}
\usepackage[utf8]{inputenc}

\usepackage{amsmath}
\usepackage{amssymb}
\usepackage{xcolor}
\usepackage{amsthm}
\usepackage[mathcal]{euscript}

\usepackage{url}

\newcommand{\Hcal}{\mathcal{H}}
\newcommand{\real}{\mathbb{R}}
\newcommand{\interior}[1]{%
  {\kern0pt#1}^{\mathrm{o}}%
}
\newcommand{\Tcal}{\mathcal{T}}
\newcommand{\Lcal}{\mathcal{L}}

\title{Notes on Hahn-Banach}
\author{Nazarov Ivan}

\date{\today}

\begin{document}
\maketitle

Consider a Banach space $(\Hcal, \|\cdot\|)$ with the norm $\Tcal^{\|\cdot\|}_\Hcal$
topology on it.

\paragraph{Linear spans} % (fold)
\label{par:linear_spans}

Let $M \subset \Hcal$ be a linear subspace of $(\Hcal, \|\cdot\|)$ and $x \notin M$.
Observe that since $M$ is a linear space $0\in M$, whence $x \neq 0$. Define $M_1$
as the linear span of $M_0 \cup \{x\}$: $M_1 = \{\alpha x + y \colon \alpha \in \real
\,,\, y \in M_0\}$.

For any $z\in M_1$ we have $z = \alpha x + y$ for some $\alpha\in \real$ and $y\in M$.
Suppose there are $(\alpha_1, y_1)$ and $(\alpha_2, y_2)$ such that
\begin{equation*}
  \alpha_1 x + y_1 = \alpha_2 x + y_2
  \Leftrightarrow
    y_1 - y_2 = (\alpha_2 - \alpha_1) x
    \,,
\end{equation*}
If $\alpha_1 = \alpha_2$, then $y_1 - y_2 = 0$ and thus $y_1 = y_2$. If $\alpha_1
\neq \alpha_2$, then we proceed by contradiction: dividing both sides by the difference
yields $x = -\tfrac{y_1 - y_2}{\alpha_1 -\alpha_2}$, which implies $x\in M$, since
$M$ is a linear subspace. Hence for every $z\in M_1$ there is a unique decomposition
$z = \alpha x + y$ with $\alpha\in\real$ and $y\in M$.

We can define a pair of maps $\alpha\colon M_1\to \real$ and $y\colon M_1 \to M$,
that are given, respectively, by the $\alpha\in\real$ and $y\in M$ components of
the decomposition $z = \alpha x + y$ for each $z\in M_1$. These maps are injections,
since $\alpha(z_1) = \alpha(z_2)$ and $y(z_1) = y(z_2)$ imply
\begin{equation*}
  z_1 = \alpha(z_1) x + y(z_1)
    = \alpha(z_2) x + y(z_2)
    = z_2
    \,.
\end{equation*}
By definition of $M_1$ the maps are also surjective. It is noteworthy that they are
also linear: for $u, v\in M_1$ and $\beta \in \real$
\begin{align*}
  \alpha(u + \beta v) x + y(u + \beta v)
    &= u + \beta v
    = u + \beta (v)
    \\
    &= \alpha(u) x + y(u)
      + \beta \bigl(\alpha(v) x + y(v) \bigr)
    \\
    &= \bigl( \alpha(u) + \beta\alpha(v) \bigr) x
    + \bigl( y(u) + \beta y(v) \bigr)
    \,,
\end{align*}
from which uniqueness of the decomposition implies linearity. Finally $M\subset M_1$,
since $y = 0 x + y \in M_1$ for any $y\in M$.

Therefore, the linear subspace $M_1$ is homomorphic to $\real \times M$ -- a linear
product space with addition and scaling over the field $\real$ defined naturally.
One can also easily make it into a normed space, by additively composing the norms
on $\real$ and $M$. And if $M$ was complete, then $M_1$ would also be complete with
respect to the newly defined norm.

% paragraph linear_spans (end)

\paragraph{Simple one dim extension} % (fold)
\label{par:simple_one_dim_extension}

Let $M_0\subseteq \Hcal$ be a linear subspace of $(\Hcal, \|\cdot\|)$, and $l$ be
a linear functional $M_0\to \real$.

For a given arbitrary $x_1\notin M_0$ let $M_1 = \{\alpha x_1 + y \colon \alpha \in
\real\,,\, y \in M_0\}$. Hence there exist linear bijections $\alpha\colon M_1\to
\real$ and $y_0\colon M_1\to M_0$ such that $z = \alpha(z) x_1 + y_0(z)$ for all
$z\in M_1$.

Define $\tilde{l}\colon M_1 \to \real$ as follows: for some $c \in \real$ for any 
$z \in M_1$ put
\begin{equation*}
  \tilde{l}(z; c)
    = \tilde{l}\bigl(\alpha(z) x_1 + y_0(z) \bigr)
    = \alpha(z) \tilde{l}(x_1) + l(y_0(z))
    = \alpha(z) c + l(y_0(z))
    \,.
\end{equation*}
Note that $\tilde{l}(\cdot; c)$ inherits linearity from the maps $\alpha$ and $y_0$.
Also $\alpha(z) = 0$ and $y_0(z) = z$ on $z\in M_0$ imply that $\tilde{l}(\cdot; c)%
\big\vert_{M_0} = l$.

Thus $\tilde{l}(\cdot; c)$ for any $c\in \real$ is an extension of $l$ from $M_0
\subset \Hcal$ to the linear span of $M_0 \cup \{x_1\}$ for an $x_1\notin M_0$.

Note that the map $c\mapsto \tilde{l}(\cdot; c)$ from $\real$ to $L_{M_0}(M_1, \real)$,
i.e. the set of all extensions of $l$ to $M_1$, is a bijection. Indeed, consider some
other linear functional $f\colon M_1 \to \real$ with $f\big\vert_{M_0} = l$. Then
$\tilde{l}\bigl(\cdot; f(x_1)\bigr) = f$, since
\begin{equation*}
  f\bigl(\alpha x_1 + y \bigr)
    = \alpha f(x_1) + f(y)
    = \alpha f(x_1) + l(y)
    = \tilde{l}\bigl(\alpha x_1 + y; f(x_1)\bigr)
    \,.
\end{equation*}
For injectivity suppose $\tilde{l}(\cdot; c_1) = \tilde{l}(\cdot; c_2)$. This implies
that $\alpha (c_1 - c_2) + l(y - y) = 0$ for any $\alpha \in \real$ and $y\in M_0$.
In particular for $\alpha = c_1 - c_2$ we get $(c_1 - c_2)^2 = 0$, whence $c_1 = c_2$.

% paragraph simple_one_dim_extension (end)

\paragraph{Extension with continuity} % (fold)
\label{par:extension_with_continuity}

Let $M_0 \subseteq \Hcal$ be a linear subspace and $l\in M_0^*$ be a continuous
linear functional on $M_0$ with $\|l\|_{M_0^*} = 1$, where
\begin{equation*}
  \|l\|_{M_0^*}
    = \sup\bigl\{\lvert l(z) \rvert\colon
        z\in M_0\,,
        \,\|z\|\leq 1
      \bigr\}
    \,.
\end{equation*}
Note that if $\|l\|_{M_0^*} \neq 1$ is nonzero and finite then we define $\hat{l}$
as $\tfrac1{\|l\|_{M_0^*}} l$, and work with that.

Pick and $x_0\notin M_0$ and let $\tilde{l}(\cdot; c)$ be an extension of $l$ from
$M_0\subset \Hcal$ to $M_1 = \{\alpha x + y \colon \alpha \in \real\,,\, y \in M_0\}$.
Since the any such extension is uniquely identified by $c$, i.e. its value on $x_1$,
by carefully chosing $c$ we shall try to make $\tilde{l} = \tilde{l}(\cdot; c)$ a
bounded linear functional.

First, note that the definition of the norm of a linear functional (operator) $\phi
\colon M\to \real$ implies that $\tfrac{\lvert \phi(z) \rvert}{\|z\|} \leq \|\phi\|_{M^*}$
for any $z\in M$ with $z \neq 0$, and $\|\phi\|_{M^*}$ is the smallest $K\in [0,
+\infty]$ such that $\lvert \phi(z) \rvert \leq K \|z\|$ for all $z\in M$.

Take $\tilde{l}$: even if its norm is $+\infty$ we still have $\lvert \tilde{l}(z)\rvert
\leq \|\tilde{l}\|_{M_1^*} \|z\|$ for all $z\in M_1$. In particular, for $z\in M_0$
we get $\lvert l(z)\rvert = \lvert \tilde{l}(z)\rvert \leq \|\tilde{l}\| \|z\|$,
whence $\|l\|_{M_0^*} \leq \|\tilde{l}\|_{M_1^*}$.

To get an upper bound on $\|\tilde{l}\|_{M_1^*}$, note the equivalence
\begin{align*}
  &\lvert \tilde{l}(z) \rvert \leq \|z\|
      \quad \forall z \in M_1
  \\ &\Leftrightarrow
    - \|\alpha x_1 + y\|
      \leq \alpha c + l(y)
      \leq \|\alpha x_1 + y\|
      \quad \forall \alpha \in \real\,, y \in M_0
  \\ &\Leftrightarrow
    - \bigl\|x_1 + \tfrac{y}\alpha\bigr\|
      \leq c + l\bigl(\tfrac{y}\alpha\bigr) \leq
        \bigl\|x_1 + \tfrac{y}\alpha\bigr\|
      \quad \forall \alpha \neq 0\,, y \in M_0
  \\ &\Leftrightarrow
    - \|x_1 + y\| \leq c + l(y) \leq \|x_1 + y\|
      \quad \forall y \in M_0
    \,.
\end{align*}
The transition from $\alpha \in \real$ to $\alpha = 0$, is valid, since $\|l\|_{M_0^*}
= 1$, $z = \alpha x_1 + y \in M_0$ whenever $\alpha = 0$, and $\tilde{l}\big\vert_{M_0}
= l$. Dividing by $\alpha \neq 0$ effectively keeps the inequality signs intact, which
can be seen by considering cases $\alpha > 0$ and $\alpha < 0$ separately. Final
equivalence, stems $M_0$'s being a linear space.

So if we could find such $c$ that for every $y \in M_0$
\begin{equation*}
  - \|x_1 + y\| - l(y) \leq c \leq \|x_1 + y\| - l(y)
    \,,
\end{equation*}
then $\|\tilde{l}\|_{M_1^*} \leq 1$.

For any $y_1, y_2 \in M_0$ we have
\begin{align*}
  l(y_1) - l(y_2)
    % = l(y_1 - y_2)
    &\leq \|l\|_{M_0^*} \|y_1 - y_2\|
    % \leq \|l\|_{M_0^*} \|y_1 + x_1 - (y_2 + x_1)\|
    \leq \|y_1 + x_1 \| + \|x_1 + y_2\|
    \,, \\
  &\Leftrightarrow
  - \|x_1 + y_2\| - l(y_2)
    \leq \|y_1 + x_1 \| - l(y_1)
    \,.
\end{align*}
This implies that the left-hand side for any $y_2\in M_0$ is bounded by the right-%
hand side at any given $y_1\in M_0$, which means that for
\begin{equation*}
  c_-
    = \sup_{y_2\in M_0} - \|x_1 + y_2\| - l(y_2)
    \,\text{ and }\,
    c_+ = \inf_{y_1\in M_0} \|y_1 + x_1 \| - l(y_1)
    \,,
\end{equation*}
we have $c_- \leq c_+$ and that $[c_-, c_+]$ is nonempty. If $c\in [c_-, c_+]$, then
\begin{equation*}
  - \|x_1 + y\| - l(y) \leq c_- \leq c \leq c_+ \leq \|x_1 + y\| - l(y)
    \,,
\end{equation*}
for every $y \in M_0$. Therefore $1 = \|l\|_{M_0^*} \leq \|\tilde{l}\|_{M_1^*} \leq 1$, 
and $\tilde{l} = \tilde{l}(\cdot; c)$ is a bounded linear functional on $M_0$ with
$\tilde{l}\big\vert_{M_0} = l$ and $\|l\|_{M_0^*}=\|\tilde{l}\|_{M_1^*}$.

% paragraph extension_with_continuity (end)

\paragraph{Hahn-Banach point-set separation} % (fold)
\label{par:hahn_banach_point_set_separation}

Let $C$ be a nonempty, closed, convex set with $0\in \interior{C}$. Then, for any
$x_0\notin C$ there is $l_{x_0} \in \Hcal^*$, a continuous linear functional on
$\Hcal$, such that $l_{x_0}(x) \leq 1 < l_{x_0}(x_0)$ for $x\in C$.

Let $p$ be the Minkowski functional of the set $C$, and consider a linear subspace
$\Lcal_{x_0} = \{t x_0\colon t\in \real\}$. $\Lcal_{x_0}$ is nontrivial, since $0\in C$
and $x_0\notin C$. Observe that for a given $x\in \Lcal_{x_0}$ there is a unique
$t_x \in \real$ such that $x = t_x x_0$. Indeed, $t_1\neq t_2$ with $t_1 x_0 = t_2 x_0$
would imply $x_0 = 0$.

The map $\tau\colon \Lcal_{x_0} \to \real$ defined by $\tau(x) = t_x$ is one-to-one
and linear. Indeed, if $x, y\in \Lcal_{x_0}$, then $\tau(x) = \tau(y)$ implies that
$x = \tau(x) x_0 = \tau(y) x_0 = y$, and for any $\alpha\in \real$
\begin{equation}
  \bigl( \tau(x) x_0 \bigr) + \bigl( \tau(y) x_0 \bigr) \alpha
    % structure of $x$ and $y$
    = x + y \alpha
    % linearity of $\Lcal_{x_0}$
    = \tau(x + y \alpha) x_0
    % uniquenes of x\mapsto t_x
  \,.
\end{equation}

Since $x_0 \notin C$ and $C$ is closed, we must have $x_0\notin \{p\leq 1\}$, whence
$p(x_0) > 1$. Let $f(x) = \tau(x) p(x_0)$ for any $x \in \Lcal_{x_0}$. On $\Lcal_{x_0}$
we have $f\leq p$ since $\tau(x) p(x_0) \leq p\bigl(\tau(x) x_0\bigr) = p(x)$ by
sublinearity of $p$, which follows from the {\bf nice} properties of $C$. At the same
time $f$ inherits linearity from $\tau$.

By the {\it Hahn-Banach Extension Theorem} there must be linear extension $l_{x_0}$
of $f$ from $\Lcal_{x_0}$ onto $\Hcal$ with $l_{x_0}\big\vert_{\Lcal_{x_0}} = f$ and
$l_{x_0} \leq p$ everywhere on $\Hcal$. This extension is a bounded linear functional
by an earlier result, and satisfies $l_{x_0}(x) \leq p(x) \leq 1$ for any $x\in C$
and $1 < p(x_0) = \tau(x_0) p(x_0) = f(x_0) = l_{x_0}(x_0)$.

% paragraph hahn_banach_point_set_separation (end)

\end{document}
