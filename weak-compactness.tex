\documentclass[a4paper]{article}

\usepackage[T1]{fontenc}
\usepackage[utf8]{inputenc}

\usepackage{amsmath}
\usepackage{amssymb}
\usepackage{xcolor}
\usepackage{amsthm}
\usepackage[mathcal]{euscript}

\usepackage{url}

\newcommand{\Hcal}{\mathcal{H}}
\newcommand{\real}{\mathbb{R}}

\title{Study of the constructive proof of weak compactness}
\author{Nazarov Ivan}

\date{\today}

\begin{document}
\maketitle

Let $(\Hcal, \langle\cdot,\cdot\rangle)$ be a Hilbert space. A sequence $(x_n)_{n\geq1} 
\in \Hcal$ converges {\bf strongly} to $x\in\Hcal$ if $\|x_n - x\| \to 0$, and {\bf
weakly} if $\langle x_n, z \rangle \to \langle x, z\rangle$ for all $z\in \Hcal$.
Weak convergence is necessary for strong convergence, since for every $z\in \Hcal$
by the Cauchy-Schwartz inequality
\begin{equation*}
  \liminf_{n\to \infty}
    \lvert \langle x_n - x, z\rangle \rvert
    \leq \limsup_{n\to \infty}
      \lvert \langle x_n - x, z\rangle \rvert
    \leq \|z\| \limsup_{n\to \infty} \|x_n - x \|
    = 0
    \,.
\end{equation*}
Let $(x_n)_{n\geq1} \in \Hcal$ be a bounded sequence $M = \sup_{n\geq1} \|x_n\| < +\infty$.
Then there is $(n_k)_{k\geq1}\uparrow$ such that $x_{n_k}$ converges weakly to some
$h \in \Hcal$, i.e. $x_{n_k}\rightharpoonup h$.

\paragraph{Proof} % (fold)
\label{par:proof}

Let $\Hcal^0_0 = \{\sum_{i=1}^n \beta_i x_i \colon \beta_i \in \real,\,n\geq 0 \}$.
The set $\Hcal^0_0$ is a linear subspace of $\Hcal$. It is a pre-Hilbert space with
respect to $\langle\cdot, \cdot \rangle$, since it can be completed w.r.t. this inner
product to get $\Hcal_0$ and $\Hcal^0_0$ is dense in its completion.

\paragraph{Subsequence} % (fold)
\label{par:subsequence}

Let's construct a subsequence, which could potentially weakly converge to something.
Let $x^0_n = x_n$. Suppose for $k\geq 0$ we have $(x^k_n)_{n\geq 1} \subseteq (x_n)_{n\geq 1}$
and $(\alpha_i)_{i\leq k}$, such that $\langle x_i, x^k_n\rangle \to \alpha_i$ for
all $1\leq i\leq k$ as $n\to \infty$.

A sequence $\alpha^{k+1}_n = \langle x_{k+1}, x^k_n \rangle$ in $\real$ is bounded,
since $\lvert\alpha^{k+1}_n\rvert \leq \|x_{k+1}\| \|x^k_n\| \leq M^2$ and $x^k_n \in
\{x_n\colon n\geq 1\}$. Therefore, it contains a subsequence $(n_p)_{p\geq1}$ such that
$\alpha^{k+1}_{n_p}$ converges to some $\alpha^{k+1} \in \real$. If we let $x^{k+1}_p
= x^k_{n_p}$, then
\begin{itemize}
  \item $\langle x_j, x^{k+1}_p \rangle \to \alpha^j$ since $(x^{k+1}_p)_{p\geq 1}
  \subseteq (x^k_p)_{p\geq 1} \subseteq (x^j_p)_{p\geq 1}$ for $j\leq k$, and
  subsequences converge to the same limit as the parent sequence;
  \item $\langle x_{k+1}, x^{k+1}_p \rangle = \alpha^{k+1}_{n_p} \to \alpha^{k+1}$
  by construction.
\end{itemize}
For the diagonal sequence $(x^p_p)_{p\geq1}$ there are indices $(n_p)_{p\geq1} \uparrow$
such that $x^p_p = x_{n_p}$, and we have $(x^p_p)_{p\geq1} \subseteq (x^k_p)_{p\geq1}$
for all $k\geq 0$, whence we must have  $\langle x_n, x^p_p \rangle \to \alpha^n$
for all $n\geq 1$.

% paragraph subsequence (end)

\paragraph{A mapping} % (fold)
\label{par:a_mapping}

We shall show that a map $l(x) = \lim_{p\to \infty} \langle x, x_{n_p} \rangle$ is
well defined.

Right from the start we know that $l(x_n) = \alpha^n$ for all $n\geq 1$. Since the
inner product is bilinear and the limit is additive, for any $z = \sum_{i=1}^n \beta_i x_i$
we have
\begin{equation*}
  l(z)
    = \lim_{p\to \infty} \Bigl \langle \sum_{i=1}^n \beta_i x_i, x_{n_p} \Bigr \rangle
    = \sum_{i=1}^n \beta_i \lim_{p\to \infty} \langle x_i, x_{n_p} \rangle
    = \sum_{i=1}^n \beta_i l(x_i)
    = \sum_{i=1}^n \beta_i \alpha^i
    \in \real
    \,.
\end{equation*}
Therefore $l$ is defined on $\Hcal^0_0$.

Since $\Hcal^0_0$ is dense in $\Hcal_0$ w.r.t the induced norm, for any $z\in \Hcal_0$
there is $(z_k)_{k\geq1} \in \Hcal^0_0$ such that $\|z_k - z\|\to 0$. Thus for any
$\varepsilon > 0$ there is $n_\varepsilon \geq 1$ such that for all $n\geq n_\varepsilon$
we have $\|z_k - z\| \leq \tfrac\varepsilon{M}$. Hence for every $p\geq 1$ we get
\begin{align*}
  \bigl\lvert \langle z, x_{n_p} \rangle -  \langle z_k, x_{n_p} \rangle \bigr\rvert
    \leq \| z - z_k \| \|x_{n_p} \|
    \leq \tfrac\varepsilon{M} M
    \,.
\end{align*}
We conclude that $\sup_{p\geq 1} \lvert \langle z - z_k, x_{n_p} \rangle \rvert \to 0$
as $k\to \infty$, since for any $\varepsilon > 0$ there is $k_\varepsilon \geq 1$ such
that $\sup_{p\geq 1} \lvert \langle z - z_k, x_{n_p} \rangle \rvert \leq \varepsilon$ for
all $k\geq k_\varepsilon$.

Next observe that for any $k \geq j$ and all $p$ we have
\begin{align*}
  \lvert l(z_k) - l(z_j) \rvert
    &\leq \lvert l(z_j) - \langle z_j, x_{n_p} \rangle \rvert
      + \lvert l(z_k) - \langle z_k, x_{n_p} \rangle \rvert
    \\
    &+ \lvert \langle z_j - z, x_{n_p} \rangle \rvert
      + \lvert \langle z_k - z, x_{n_p} \rangle \rvert
      \,.
\end{align*}
From the convergence of the supremum above we can pick $k_\varepsilon \geq1$ such
that the last two terms are bounded each by $\tfrac\varepsilon2$ for all $j,k \geq
k_\varepsilon$. Taking limit suprema of both sides with respect to $p$ eliminates
the first two absolute terms of the right hand side, thereby implying that
$\lvert l(z_k) - l(z_j) \rvert \leq \varepsilon$ for all such $j$ and $k$. Hence
$(l(z_k))_{k\geq 1}$ is Cauchy in $\real$, and thus $l(z_k) \to \alpha$ for some
$\alpha \in \real$.

To show that $l(z) = \alpha$ we make the following observation: for all $k,p\geq 1$
\begin{equation*}
  \lvert \langle z, x_{n_p} \rangle - \alpha \rvert
    \leq \lvert l(z_k) - \alpha \rvert
    + \lvert \langle z_k, x_{n_p} \rangle - l(z_k) \rvert
    + \sup_{p\geq 1} \lvert \langle z - z_k, x_{n_p} \rangle \rvert
    \,.
\end{equation*}
For any $\varepsilon > 0$ there is $k_\varepsilon \geq 1$ such that the sum of the
first and last terms on the right hand side is not greater than $\varepsilon$ at
$k_\varepsilon$. Hence the $\limsup_{p\to \infty}$ of the expression in the left
is not greater than $\varepsilon$, since
\begin{equation*}
  \limsup_{p\to \infty} \lvert \langle z, x_{n_p} \rangle - \alpha \rvert
    \leq \varepsilon
    + \limsup_{p\to \infty} \lvert
        \langle z_{k_\varepsilon}, x_{n_p} \rangle - l(z_{k_\varepsilon})
      \rvert
    \,,
\end{equation*}
and $\langle z_k, x_{n_p} \rangle \to l(z_k)$ in $\real$ as $p\to \infty$ for any
$k \geq 1$. Since the last bound holds for arbitrarily small $\varepsilon > 0$,
$\langle z, x_{n_p} \rangle \to \alpha$, and therefore $l$ is defined on $\Hcal_0$.

What about $z\in \Hcal_0^\perp$, the orthogonal complement of $\Hcal_0$ in $\Hcal$?
For any $z\in \Hcal_0^\perp$ by orthogonality $\langle z, x_n \rangle = 0$ for all
$n\geq 1$, whence $\lim_{p\to\infty} \langle z, x_{n_p} \rangle = 0$. Therefore $l$
is defined on $\Hcal_0^\perp$.

% paragraph a_mapping (end)

\paragraph{Linearity and continuity} % (fold)
\label{par:linearity_and_continuity}
 
Thus $z\mapsto l(z) = \lim_{p\to\infty} \langle z, x_{n_p} \rangle$ is a well defined
$\Hcal \to \real$ map. Furthermore, it is linear, since the inner product is bilinear:
\begin{equation*}
  l(z + \alpha x)
    = \lim_{p\to\infty} \langle z + \alpha x, x_{n_p} \rangle
    = \lim_{p\to\infty} \langle z, x_{n_p} \rangle
    + \alpha \lim_{p\to\infty} \langle x, x_{n_p} \rangle
    = l(z) + \alpha l(x)
    \,.
\end{equation*}
Finally, $l$ is also a bounded map, since for all $z\in \Hcal$
\begin{equation*}
  \lvert l(z) \rvert
    = \lim_{p\to\infty} \lvert \langle z, x_{n_p} \rangle \rvert
    \leq \sup_{p\geq 1} \|z\| \| x_{n_p}\|
    \leq M \|z\|
    \,.
\end{equation*}
So the map $l$ is linear and bounded, hence continuous.

% paragraph linearity_and_continuity (end)

\paragraph{Weak limit} % (fold)
\label{par:weak_limit}

Riesz representer theorem therefore implies the existence of some $h\in \Hcal$ such
that $l(z) = \langle z, h\rangle$ for any $z\in \Hcal$. Therefore $x_{n_p} \rightharpoonup
h$ for this $(x_{n_p})_{p\geq 1} \subseteq (x_n)_{n\geq 1}$.

% paragraph weak_limit (end)

% paragraph proof (end)

\end{document}
