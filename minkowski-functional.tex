\documentclass[a4paper]{article}

\usepackage[T1]{fontenc}
\usepackage[utf8]{inputenc}

\usepackage{amsmath}
\usepackage{amssymb}
\usepackage{xcolor}
\usepackage{amsthm}
\usepackage[mathcal]{euscript}

\usepackage{url}

\newcommand{\Hcal}{\mathcal{H}}
\newcommand{\real}{\mathbb{R}}
\newcommand{\inter}{{\mathtt{int}}}

\title{Notes on Minkowski Functional and Hahn-Banach}
\author{Nazarov Ivan}

\date{\today}

\begin{document}
\maketitle

Cosnsider a Banach space $(\Hcal, \|\cdot\|)$. For any $C \subseteq \Hcal$ define
the {\it Minkowski Functional} $p\colon \Hcal \to [0, +\infty]$ by
\begin{equation*}
  p(x)
    = \inf \bigl\{ \lambda > 0\,:\, x \in \lambda C\bigr\}
  \,,
\end{equation*}
where $\lambda C = \{\lambda z \,:\, z\in C\}$ is the $\lambda$-inflation of $C$,
$\lambda \in\real$. This functional has many useful properties when $C$ is convex
and the interior of $C$ contains $0$:
\begin{itemize}
  \item $p$ is real-valued, positively homogeneous, continuous and subadditive
  \item $p^{-1}\bigl([0, 1]\bigr) = C$ if $C$ is closed in $\Hcal$
\end{itemize}

\paragraph{Properties of $\lambda$-inflation} % (fold)
\label{par:properties_of_lambda_inflation}

Consider any $C\subseteq \Hcal$, any $t \neq 0$ and any $x\in \Hcal$. Then $x\in t C$
means that there exists $z \in C$ such that $x = t z$. Therefore $\tfrac1t x = z
\in C$, since $\Hcal$ is a vector space. For such $(C, x, t)$ we have the implication
$x \in t C \Rightarrow \tfrac1t x \in C$.

We get the reverse implication by applying the forward one above to the tuple $(C, x, t)
= (\alpha K, \tfrac1\alpha z, \tfrac1\alpha)$, noting that $z = \alpha x$. Therefore
$x\in t C$ iff $\tfrac1t x \in C$ for any $C\subseteq \Hcal$, $x\in \Hcal$ and $t > 0$.
Furthermore, for any $\alpha, \beta \neq 0$ we have
\begin{equation*}
  \alpha x \in \beta C \Leftrightarrow
  \beta^{-1} \alpha x \in C \Leftrightarrow
  \beta^{-1} x \in \alpha^{-1} C \Leftrightarrow
  x \in \beta \alpha^{-1} C
  \,.
\end{equation*}

Suppose $C$ is convex, then $\beta C$ is convex for any $\beta$. Indeed, for $x_1,
x_0 \in \beta C$ and $\theta\in [0, 1]$ there are $z_1, z_0 \in C$ such that $x_i
= \beta z_i$. But then $x_\theta \in \beta C$, since we have $z_\theta = \theta z_1
+ (1-\theta) z_0 \in C$ and
\begin{equation*}
  x_\theta
  = \theta x_1 + (1-\theta) x_0
  = \theta \beta z_1 + (1-\theta) \beta z_0
  = \beta z_\theta
  \,.
\end{equation*}

Suppose $C$ is convex and $0\in C$. Then for any $\theta \in (0, 1)$ and $x\in C$
we have $\theta x = (1 - \theta)\, 0 + \theta x \in C$, hence $x \in \theta^{-1} C$.
Therefore $C \subseteq \lambda C$ for all $\lambda \geq 1$. Next, if $\alpha > \beta$,
then the $\beta$-inflation of $C$ is convex, and $0 \in \beta C$. Therefore $\beta C
\subseteq \tfrac\alpha\beta (\beta C) = \alpha C$.

% paragraph properties_of_lambda_inflation (end)

\paragraph{Zero} % (fold)
\label{par:zero}

The Minkowski functional is zero on $0$ if $0\in C$. Indeed, for any $\lambda > 0$
we have $0 \in \lambda C$, since $0 = \lambda 0$. Therefore, $p(0) \leq \lambda$
for all $\lambda > 0$, whence $p(0) = 0$.

% paragraph zero (end)

\paragraph{Positive homogeneity} % (fold)
\label{par:pos_homogeneity}

Let $t > 0$. If $x\in\Hcal$ is such that $p(x) = +\infty$, then $x\notin \lambda C$
for any $\lambda>0$. In particular $t x\notin t \tfrac\lambda{t} C$, whence $p(tx)
= +\infty$.

If $x \in \Hcal$ such that $p(x) < +\infty$, then for any $U > p(x)$ there exists
$\lambda < U$ with $x\in \lambda C$. Hence $t x \in t \lambda C$ and $p(t x) \leq
t \lambda < t U$. Therefore $p(t x) < +\infty$ and $p(t x) \leq t p(x)$.

The opposite inequality follows from applying the direct one to $(x, t) = (\alpha
z, \alpha^{-1})$ for any $\alpha > 0$ and $z \in \Hcal$. Indeed, since $\alpha^{-1}
x = z$, $p(\alpha^{-1} x) \leq \alpha^{-1} p(x)$ implies $\alpha p(z) \leq p(\alpha z)$.
Therefore $p(t x) = t p(x)$ for any $x \in \Hcal$ and every $t > 0$.

% paragraph pos_homogeneity (end)

\paragraph{Homogeneity via balancedness} % (fold)
\label{par:homogeneity_via_balancedness}

If $C$ is balanced then $p(x)$ is homogenous.

A set is $C$ is balanced if $\beta C \subset C$ for all $\lvert \beta \rvert \leq 1$.
For any balanced $C$ we have $\lambda C \subset C$ in particular for $\lambda = 0$,
whence $\{0\} = 0 C \subset C$. Note that for any $\alpha \neq 0$ the set $\alpha C$
is balanced as well. Indeed, we have $\beta (\alpha C) = \alpha (\beta C) \subset
\alpha C$.

Suppose $\alpha x \in \lambda C$ for $\alpha \neq 0$ and $\lambda > 0$. Then we have
\begin{equation*}
  \lvert \alpha \rvert x
    = \tfrac{\lvert \alpha \rvert}\alpha (\alpha x)
    \in \tfrac{\lvert \alpha \rvert}\alpha (\lambda C)
    = \lambda \Bigl( \tfrac{\lvert \alpha \rvert}\alpha C \Bigr)
    \subset \lambda C
    \subseteq \lambda C
    \,,
\end{equation*}
by definition of the set inflation, vector space $\Hcal$ and from balancedness of
$C$. Conversely $\lvert \alpha \rvert x \in \lambda C$ implies $\alpha x \in \lambda
C$:
\begin{equation*}
  \alpha x
    = \tfrac\alpha{\lvert \alpha \rvert} (\lvert \alpha \rvert x)
    \in \tfrac\alpha{\lvert \alpha \rvert} (\lambda C)
    = \lambda \Bigl( \tfrac\alpha{\lvert \alpha \rvert} C \Bigr)
    \subset \lambda C
    \subseteq \lambda C
    \,.
\end{equation*}
For $t = 0$ we have $0 = p(0 x) = 0 p(x) = 0$ and for $t\neq 0$ we consider the
following:
\begin{equation*}
  p(t x)
    = \inf\{\lambda > 0\colon t x \in \lambda C\}
    = \inf\{\lambda > 0\colon \lvert t \rvert x \in \lambda C\}
    = p\bigl( \lvert t \rvert x \bigr) = \lvert t \rvert p(x)
    \,.
\end{equation*}

% paragraph homogeneity_via_balancedness (end)

\paragraph{Subadditivity} % (fold)
\label{par:subadditivity}

We claim that $p$ is subadditive if $C$ is convex.

Consider $x_1, x_2 \in \Hcal$. If $p(x_i) = +\infty$, then trivially $p\bigl( x_1
+ x_2 \bigr) \leq +\infty =  p(x_1) + p(x_2)$. Suppose $p(x_i) < + \infty$. Then
for any $\varepsilon > 0$ there are $\lambda_i > 0$ with $x_i \in \lambda_i C$ such
that $\lambda_i < p(x_i) + \tfrac12\varepsilon$ for any $i$. We argue that $x_1 + x_2
\in (\lambda_1 + \lambda_2) C$, since $C$ is convex, $\tfrac{x_i}{\lambda_i} \in C$,
and
\begin{equation*}
  \tfrac1{\lambda_1 + \lambda_2} (x_1 + x_2)
    = \tfrac{\lambda_1}{\lambda_1 + \lambda_2} \frac{x_1}{\lambda_1}
      + \tfrac{\lambda_2}{\lambda_1 + \lambda_2} \frac{x_2}{\lambda_2}
    \,.
\end{equation*}
Therefore $p\bigl( x_1 + x_2 \bigr) \leq p(x_1) + p(x_2)$, since for all $\varepsilon
> 0$
\begin{equation*}
  p\bigl( x_1 + x_2 \bigr)
    \leq \lambda_1 + \lambda_2
    < p(x_1) + p(x_2) + \varepsilon
    \,.
\end{equation*}

% paragraph subadditivity (end)

\paragraph{Sublinearity} % (fold)
\label{par:sublinearity}

If $C$ is convex and balanced (or just $0 \in C$), then we have $p(t x) = t p(x)$
for any $t\geq 0$ and $x\in \Hcal$ such that $p(x) < +\infty$. This implies that
$p(-t x)$ is finite when $p(-x)$ is finite. Hence is $p(x), p(-x) < +\infty$ then
for $t \leq 0$
\begin{equation*}
  0 = p(0)
    = p\bigl( t x + (- t) x \bigr)
    \leq p\bigl( t x \bigr) + p\bigl( (- t) x \bigr)
    = p\bigl( t x \bigr) + (- t) p( x )
    \,,
\end{equation*}
and therefore $t p(x) \leq p(t x)$ for any $t\in \real$.

% paragraph sublinearity (end)

\paragraph{the sufficient condition for finite values} % (fold)
\label{par:the_sufficient_condition_for_finite_values}

Obviously, if the set $\{ \lambda > 0 \colon x \in \lambda C \}$ is non-empty for
any $x \in \Hcal$, then $p(x)$ is finite-valued, because every $x$ is covered by
at least one $\lambda$-inflation.

This condition implies that $0\in C$. Indeed, under it there must be $\lambda > 0$
such that $0 \in \lambda C$, which in turn means that $\exists z\in C$ such that
$0 = \lambda z$. Since $\lambda > 0$, we get $z\in C$.

So $0\in C$ seems necessary for finiteness. However, it is not sufficient. Consider
a cone $C = \{t z\colon z\in A\} = \bigcup_{t\geq 0} t A$ for $A = \{z\colon \|z - a\|
\leq \|a\|\}$ for a given $a \neq 0$. Since $t A = \{z\colon \|z - t a\| \leq t \|a\| \}$
for any $t\geq 0$, we still have $-a \notin C$: if it were otherwise, then $\tfrac{- a}t
\in A$ for some $t > 0$ with $1 + \tfrac1t \leq 1$ --- a contradiction.

With this counterexample in mind we can formulate the following conjecture: $p(\Hcal)
\subseteq \real$ it is sufficient that $0$ be in the topological interior of $C$,
i.e. the largest open set thats fits inside $C$. Indeed, in this case there would
be a $\varepsilon > 0$, such that $B(0, \varepsilon) = \{z\colon \|z\| < \varepsilon\}
\subseteq C$, implying that for a given $\eta \in (0, 1)$ we have
\begin{equation*}
  \tfrac{\eta}{\|x\|} x \in B(0, 1) = \varepsilon^{-1} B(0, \varepsilon)
  \Leftrightarrow
  \tfrac{\varepsilon \eta}{\|x\|} x \in B(0, \varepsilon)
  \subseteq C
  \Leftrightarrow
  x \in \tfrac{\|x\|}{\eta \varepsilon} C
  \,,
\end{equation*}
for any $x\neq 0$. Therefore $p(x) < +\infty$ for any $x\in \Hcal$ for such $C$.

% paragraph the_sufficient_condition_for_finite_values (end)

\paragraph{Continuity} % (fold)
\label{par:continuity}

We claim that for $p\colon \Hcal \to [0, +\infty]$ to be continuous, it suffices
for $C$ to have $0$ in its interior, be closed and convex. Note that balancedness
does not imply these properties: take a union of a closed convex cone and its mirrored
counterpart.

First, note that since $C$ is convex for any $x \in \Hcal$ and $h\in V$ we have
\begin{equation*}
  % p(x)
  %   \leq p(x+h) + p(-h)
  %   \,, \text{ and }
  % p(x + h)
  %   \leq p(x) + p(h)
  %   \,,
  \bigl\lvert p(x + h) - p(x) \bigr\rvert
    \leq p(h) \vee p(-h)
    \leq \sup_{h\in V} p(h)
    \,,
\end{equation*}
where it is assumed that $h\in V$ iff $-h\in V$. If for any $x$ and any $\varepsilon
> 0$ we could find $V$ such that $\sup_{h\in V} p(h) \leq \varepsilon$.

Indeed, let $C$ be such a set and pick any $\alpha > 0$. If $x \in \alpha C$, then
that $p(x) \leq \alpha$ and thus $\alpha C \subseteq p^{-1}\bigl([0, \alpha]\bigr)$.
On the other hand, if $x\in p^{-1}\bigl([0, \alpha)\bigr)$, then $x \in \lambda_x C
\subseteq \alpha C$ for some $\lambda_x \in [p(x), \alpha)$. Therefore
\begin{equation*}
  p^{-1}\bigl([0, \alpha)\bigr)
    \subseteq \alpha C
    \subseteq p^{-1}\bigl([0, \alpha]\bigr)
    \,.
\end{equation*}
Note that $\lambda_x C \subseteq \alpha C$ for $\lambda_x < \alpha$ follows from
$0\in C$ and convexity of $C$.

There is $\delta > 0$, such that $B(0, \delta) \subseteq C$, which means that for
any $\eta \in (0, 1)$
\begin{equation*}
  0 \in  B(0, \alpha \eta \delta)
    = \alpha \eta B(0, \delta)
    \subseteq \alpha \eta C
    \subseteq p^{-1}\bigl([0, \alpha \eta] \bigr)
    \subseteq p^{-1}\bigl([0, \alpha) \bigr)
    \,,
\end{equation*}
since $[0, \alpha \eta] \subset [0, \alpha)$. Hence for $\alpha > 0$ there is $\delta
> 0$ with $B(0, \delta) \subseteq p^{-1}\bigl([0, \alpha) \bigr)$. 

\begin{align*}
  p(x)
    &\leq p(x+h) + p(-h) < p(x+h) + \alpha
    \,, \\
  p(x + h)
    &\leq p(x) + p(h) < p(x) + \alpha
    \,,
\end{align*}
Thus $B(x, \delta) \in p^{-1}\bigl(B(p(x), \alpha)\bigr)$


First of all $p(\Hcal) \subseteq \real$.


% paragraph continuity (end)

\end{document}
