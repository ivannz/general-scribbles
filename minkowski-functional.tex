\documentclass[a4paper]{article}

\usepackage[T1]{fontenc}
\usepackage[utf8]{inputenc}

\usepackage{amsmath}
\usepackage{amssymb}
\usepackage{xcolor}
\usepackage{amsthm}
\usepackage[mathcal]{euscript}

\usepackage{url}

\newcommand{\Hcal}{\mathcal{H}}
\newcommand{\real}{\mathbb{R}}
\newcommand{\interior}[1]{%
  {\kern0pt#1}^{\mathrm{o}}%
}
\newcommand{\Tcal}{\mathcal{T}}

\title{Notes on Minkowski Functional and Hahn-Banach}
\author{Nazarov Ivan}

\date{\today}

\begin{document}
\maketitle

Consider a Banach space $(\Hcal, \|\cdot\|)$ with the norm $\Tcal^{\|\cdot\|}_\Hcal$
on it $\Hcal$. In the following $[C]$ is the closure of a set $C$ in the topology
of $\Hcal$: $[C]$ the smallest closed set covering $C$, $[C] = [[C]]$, and monotone.
In contrast, $\interior{C}$ --- the topological interior of $C$ in $\Hcal$, --- is
the largest open set contained within $C$.

For any nonempty $C \subseteq \Hcal$ define the {\it Minkowski Functional} $p\colon
\Hcal \to [0, +\infty]$ by
\begin{equation*}
  p(x)
    = \inf \bigl\{ \lambda > 0\,:\, x \in \lambda C\bigr\}
    \,,
\end{equation*}
where $\lambda C = \{\lambda z \,:\, z\in C\}$ is the $\lambda$-inflation of $C$,
$\lambda \in\real$. This functional has many useful properties when $C$ is convex
and the interior of $C$ contains $0$.

\paragraph{Properties of $\lambda$-inflation} % (fold)
\label{par:properties_of_lambda_inflation}

Consider any $C\subseteq \Hcal$, any $t \neq 0$ and any $x\in \Hcal$. Then $x\in t C$
means that there exists $z \in C$ such that $x = t z$. Therefore $\tfrac1t x = z
\in C$, since $\Hcal$ is a vector space. For such $(C, x, t)$ we have the implication
$x \in t C \Rightarrow \tfrac1t x \in C$.

We get the reverse implication by applying the forward one above to the tuple $(C, x, t)
= (\alpha K, \tfrac1\alpha z, \tfrac1\alpha)$, noting that $z = \alpha x$. Therefore
$x\in t C$ iff $\tfrac1t x \in C$ for any $C\subseteq \Hcal$, $x\in \Hcal$ and $t > 0$.
Furthermore, for any $\alpha, \beta \neq 0$ we have
\begin{equation*}
  \alpha x \in \beta C \Leftrightarrow
  \beta^{-1} \alpha x \in C \Leftrightarrow
  \beta^{-1} x \in \alpha^{-1} C \Leftrightarrow
  x \in \beta \alpha^{-1} C
  \,.
\end{equation*}

Suppose $C$ is convex, then $\beta C$ is convex for any $\beta$. Indeed, for $x_1,
x_0 \in \beta C$ and $\theta\in [0, 1]$ there are $z_1, z_0 \in C$ such that $x_i
= \beta z_i$. But then $x_\theta \in \beta C$, since we have $z_\theta = \theta z_1
+ (1-\theta) z_0 \in C$ and
\begin{equation*}
  x_\theta
  = \theta x_1 + (1-\theta) x_0
  = \theta \beta z_1 + (1-\theta) \beta z_0
  = \beta z_\theta
  \,.
\end{equation*}

Suppose $C$ is convex and $0\in C$. Then for any $\theta \in (0, 1)$ and $x\in C$
we have $\theta x = (1 - \theta)\, 0 + \theta x \in C$, hence $x \in \theta^{-1} C$.
Therefore $C \subseteq \lambda C$ for all $\lambda \geq 1$. Next, if $\alpha > \beta$,
then the $\beta$-inflation of $C$ is convex, and $0 \in \beta C$. Therefore $\beta C
\subseteq \tfrac\alpha\beta (\beta C) = \alpha C$.

% paragraph properties_of_lambda_inflation (end)

\paragraph{Zero} % (fold)
\label{par:zero}

The Minkowski functional is zero on $0$ if $0\in C$. Indeed, for any $\lambda > 0$
we have $0 \in \lambda C$, since $0 = \lambda 0$. Therefore, $p(0) \leq \lambda$
for all $\lambda > 0$, whence $p(0) = 0$.

Alternatively, if $p(0) = 0$, then $0\in \lambda C$ for any $\lambda > 0$, which
means that for $\lambda=1$ there is $z\in C$ such that $z = 0$. Hence $0\in C$.

% paragraph zero (end)

\paragraph{Positive homogeneity} % (fold)
\label{par:pos_homogeneity}

Let $t > 0$. If $x\in\Hcal$ is such that $p(x) = +\infty$, then $x\notin \lambda C$
for any $\lambda>0$. In particular $t x\notin t \tfrac\lambda{t} C$, whence $p(tx)
= +\infty$.

If $x \in \Hcal$ such that $p(x) < +\infty$, then for any $U > p(x)$ there exists
$\lambda < U$ with $x\in \lambda C$. Hence $t x \in t \lambda C$ and $p(t x) \leq
t \lambda < t U$. Therefore $p(t x) < +\infty$ and $p(t x) \leq t p(x)$.

The opposite inequality follows from applying the direct one to $(x, t) = (\alpha
z, \alpha^{-1})$ for any $\alpha > 0$ and $z \in \Hcal$. Indeed, since $\alpha^{-1}
x = z$, $p(\alpha^{-1} x) \leq \alpha^{-1} p(x)$ implies $\alpha p(z) \leq p(\alpha z)$.
Therefore $p(t x) = t p(x)$ for any $x \in \Hcal$ and every $t > 0$.

By positive homogeneity we have $p(0) = p(t \cdot 0) = t p(0)$ for any $t > 0$, which
can be satisfied only if $p(0) \in \{0, \infty\}$. Hence, $p(0) = 0$ iff $0\in C$, 
and $p(0)\neq 0$ iff $0\notin C$, in which case $p(0) = +\infty$.

% paragraph pos_homogeneity (end)

\paragraph{Homogeneity via balancedness} % (fold)
\label{par:homogeneity_via_balancedness}

If $C$ is balanced then $p(x)$ is homogeneous.

A set is $C$ is balanced if $\beta C \subset C$ for all $\lvert \beta \rvert \leq 1$.
For any balanced $C$ we have $\lambda C \subset C$ in particular for $\lambda = 0$,
whence $\{0\} = 0 C \subset C$. Note that for any $\alpha \neq 0$ the set $\alpha C$
is balanced as well. Indeed, we have $\beta (\alpha C) = \alpha (\beta C) \subset
\alpha C$.

Suppose $\alpha x \in \lambda C$ for $\alpha \neq 0$ and $\lambda > 0$. Then we have
\begin{equation*}
  \lvert \alpha \rvert x
    = \tfrac{\lvert \alpha \rvert}\alpha (\alpha x)
    \in \tfrac{\lvert \alpha \rvert}\alpha (\lambda C)
    = \lambda \Bigl( \tfrac{\lvert \alpha \rvert}\alpha C \Bigr)
    \subset \lambda C
    \subseteq \lambda C
    \,,
\end{equation*}
by definition of the set inflation, vector space $\Hcal$ and from balancedness of
$C$. Conversely $\lvert \alpha \rvert x \in \lambda C$ implies $\alpha x \in \lambda
C$:
\begin{equation*}
  \alpha x
    = \tfrac\alpha{\lvert \alpha \rvert} (\lvert \alpha \rvert x)
    \in \tfrac\alpha{\lvert \alpha \rvert} (\lambda C)
    = \lambda \Bigl( \tfrac\alpha{\lvert \alpha \rvert} C \Bigr)
    \subset \lambda C
    \subseteq \lambda C
    \,.
\end{equation*}
For $t = 0$ we have $0 = p(0 x) = 0 p(x) = 0$ and for $t\neq 0$ we consider the
following:
\begin{equation*}
  p(t x)
    = \inf\{\lambda > 0\colon t x \in \lambda C\}
    = \inf\{\lambda > 0\colon \lvert t \rvert x \in \lambda C\}
    = p\bigl( \lvert t \rvert x \bigr) = \lvert t \rvert p(x)
    \,.
\end{equation*}

% paragraph homogeneity_via_balancedness (end)

\paragraph{Subadditivity} % (fold)
\label{par:subadditivity}

We claim that $p$ is subadditive if $C$ is convex.

Consider $x_1, x_2 \in \Hcal$. If $p(x_i) = +\infty$, then trivially $p\bigl( x_1
+ x_2 \bigr) \leq +\infty =  p(x_1) + p(x_2)$. Suppose $p(x_i) < + \infty$. Then
for any $\varepsilon > 0$ there are $\lambda_i > 0$ with $x_i \in \lambda_i C$ such
that $\lambda_i < p(x_i) + \tfrac12\varepsilon$ for any $i$. We argue that $x_1 + x_2
\in (\lambda_1 + \lambda_2) C$, since $C$ is convex, $\tfrac{x_i}{\lambda_i} \in C$,
and
\begin{equation*}
  \tfrac1{\lambda_1 + \lambda_2} (x_1 + x_2)
    = \tfrac{\lambda_1}{\lambda_1 + \lambda_2} \frac{x_1}{\lambda_1}
      + \tfrac{\lambda_2}{\lambda_1 + \lambda_2} \frac{x_2}{\lambda_2}
    \,.
\end{equation*}
Therefore $p\bigl( x_1 + x_2 \bigr) \leq p(x_1) + p(x_2)$, since for all $\varepsilon
> 0$
\begin{equation*}
  p\bigl( x_1 + x_2 \bigr)
    \leq \lambda_1 + \lambda_2
    < p(x_1) + p(x_2) + \varepsilon
    \,.
\end{equation*}

% paragraph subadditivity (end)

\paragraph{Sublinearity} % (fold)
\label{par:sublinearity}

If $C$ is convex and balanced (or just $0 \in C$), then we have $p(t x) = t p(x)$
for any $t\geq 0$ and $x\in \Hcal$ such that $p(x) < +\infty$. This implies that
$p(-t x)$ is finite when $p(-x)$ is finite. Hence is $p(x), p(-x) < +\infty$ then
for $t \leq 0$
\begin{equation*}
  0 = p(0)
    = p\bigl( t x + (- t) x \bigr)
    \leq p\bigl( t x \bigr) + p\bigl( (- t) x \bigr)
    = p\bigl( t x \bigr) + (- t) p( x )
    \,,
\end{equation*}
and therefore $t p(x) \leq p(t x)$ for any $t\in \real$.

% paragraph sublinearity (end)

\paragraph{the sufficient condition for finite values} % (fold)
\label{par:the_sufficient_condition_for_finite_values}

We conjecture that $p(\Hcal) \subseteq \real$ requires $0$ to be in the topological
interior of $C$, i.e. the largest open set that fits inside $C$. Indeed, in this
case there would be a $\varepsilon > 0$, such that $B(0, \varepsilon) = \{z\colon
\|z\| < \varepsilon\} \subseteq C$, implying that for a given $\eta \in (0, 1)$ we
have
\begin{equation*}
  \tfrac{\eta}{\|x\|} x \in B(0, 1) = \varepsilon^{-1} B(0, \varepsilon)
  \Leftrightarrow
  \tfrac{\varepsilon \eta}{\|x\|} x \in B(0, \varepsilon)
  \subseteq C
  \Leftrightarrow
  x \in \tfrac{\|x\|}{\eta \varepsilon} C
  \,,
\end{equation*}
for any $x\neq 0$. Therefore $p(x) < +\infty$ for any $x\in \Hcal$ for such $C$.

% paragraph the_sufficient_condition_for_finite_values (end)

\paragraph{Upper bound for continuity} % (fold)
\label{par:upper_bound_for_continuity}

Suppose $C$ is convex and such that $p(\Hcal)\subseteq \real$. Then for any $x, h
\in \Hcal$ we have $p(x) \leq p(x+h) + p(-h)$ and $p(x + h) \leq p(x) + p(h)$ by
subadditivity. Together with positive homogeneity, this implies that for any $x \in
\Hcal$ and $h\neq 0$
\begin{equation*}
  % the rhs is of the form $\|h\| \ldots$, implying that the bound holds for $h=0$
  \bigl\lvert p(x + h) - p(x) \bigr\rvert
    % \leq \max\{p(h), p(- h)\}
    \leq \| h \| \max\Bigl\{
        p\Bigl(\tfrac{h}{\|h\|}\Bigr),
        p\Bigl(- \tfrac{h}{\|h\|}\Bigr)
      \Bigr\}
    \leq \| h \| \sup \{p(u)\colon \|u\|\leq 1\}
    % \Big\vert_{u = \|h\|^{-1} h}
    \,.
\end{equation*}
Since the right-hand side does not depend on $x$, we note that the bound holds
uniformly over $x \in \Hcal$. Therefore in order to show continuity (uniform) it
is enough to prove that $p$ is bounded within $B[0, 1] = \{u\colon \|u\| \leq 1\}$
-- the closed unit ball in $\Hcal$.

% paragraph upper_bound_for_continuity (end)

\paragraph{Preimage inclusion} % (fold)
\label{par:preimage_inclusion}

Consider a convex $C$ with $0\in C$. Then for any $\alpha > 0$ we have the following.
If $x \in \alpha C$ for some $\alpha > 0$, then by definition $p(x) \leq \alpha$ and
thus $\alpha C \subseteq \{p \leq \alpha\}$. Conversely, if $x\in \{p < \alpha\}$,
then $x \in \lambda_x C$ for some $p(x) \leq \lambda_x < \alpha$. Since $C$ is convex
and $0\in C$, it must hold $\lambda_x C \subseteq \alpha C$. Therefore
\begin{equation*}
  \{p < \alpha \}
    \subseteq \alpha C
    \subseteq \{p \leq \alpha \}
    \,,
\end{equation*}
for any $\alpha > 0$. In particular we have $C \subseteq \{p \leq 1\}$ and by positive
homogeneity
\begin{equation*}
  \alpha \{p\leq 1\}
    = \{\alpha x \in \Hcal \colon p(x) \leq 1\}
    = \bigl\{z \in \Hcal \colon \tfrac1{\alpha} p(z) \leq 1\bigr\}
    = \{p \leq \alpha\}
    \,.
\end{equation*}
Consider the scaling map $x\mapsto \sigma_\alpha(x) = \alpha x$ on a vector space.
Its inverse is $\sigma_\alpha^{-1} = \sigma_{\alpha^{-1}}$, and it is easy to see
that $\sigma_\alpha$ is bouded linear map. By positive homogeneity of $p$ the maps
$\sigma_\alpha$ and $p$ commute: $\sigma_\alpha \circ p = p \circ \sigma_\alpha$
for all $\alpha > 0$. Therefore, for any $U \subseteq [0, +\infty)$ and $\alpha > 0$
we get $\{p\in \alpha U\} = \alpha \{ p\in U \}$:
\begin{align*}
    \sigma_\alpha \bigl( \{p \in U\} \bigr)
    &= \bigl\{\sigma_\alpha^{-1} \circ p \in U\bigr\}
    = \bigl\{\sigma_{\alpha^{-1}} \circ p \in U\bigr\}
    \\
    &= \bigl\{p \circ \sigma_{\alpha^{-1}} \in U\bigr\}
    = \bigl\{p \in \sigma_{\alpha^{-1}}^{-1}(U)\bigr\}
    = \bigl\{p \in \sigma_\alpha(U)\bigr\}
    \,,
\end{align*}
because for any maps $f$ and $h$ the preimages of their composition is
\begin{equation*}
  \{f\circ h \in U\}
    = \bigl\{ h \in f^{-1}(U) \bigr\}
    = h^{-1}\bigl(\{f \in U \}\bigr)
    \,.
\end{equation*}

% paragraph preimage_inclusion (end)

\paragraph{Boundedness} % (fold)
\label{par:boundedness}

Consider $C$ that has $0$ in its topological interior. This means that there is $\delta
> 0$, such that $B(0, \delta) \subseteq C$. Thus for any $\eta \in (0, 1)$
\begin{equation*}
  B[0, 1]
    = \{u\colon \|u\| \leq 1\}
    % since [0, ]
    \subseteq B\bigl( 0, \tfrac1\eta \bigr)
    % scaling and translation of balls (continuous)
    = \tfrac1{\delta \eta} B(0, \delta)
    % inflation is a monotone set-automorphism
    \subseteq \tfrac1{\delta \eta} C
    % proven fact that $\alpha C \subseteq \{p \leq \alpha\}$
    \subseteq \bigl\{p \leq \tfrac1{\delta\eta} \bigr\}
    \,,
\end{equation*}
since $[0, 1] \subset [0, \eta^{-1})$. Hence there exists $M > 0$ that depends only
on $C$, such that $p(u) \leq M$ for any $u \in B[0, 1]$. It is worth reminding that
the rightmost set inclusion follows from the definition of $p$, and does not require
convexity of $C$.

If a positively homogeneous $p$ is bounded on $\{u\colon \|u\|\leq 1\}$, then it
is finite:
\begin{equation*}
  p(x)
    \leq \|x\| p\Bigl(\tfrac{x}{\|x\|}\Bigr)
    \leq \|x\| \sup\{p(u)\colon \|u\|\leq 1\}
    < +\infty
    \,.
\end{equation*}

% paragraph boundedness (end)

\paragraph{Continuity} % (fold)
\label{par:continuity}

We claim that for $p\colon \Hcal \to [0, +\infty]$ to be continuous, it suffices
for $C$ to have $0\in \interior{C}$ and be convex. Indeed, since $0\in C$ for such
$C$ and $p$ is real-valued, the upper bounds derived earlier hold true. Hence for
any $h \in \Hcal$ with $\|h\| < \tfrac\varepsilon{M}$ and uniformly over $x$ we have
\begin{equation*}
  \bigl\lvert p(x + h) - p(x) \bigr\rvert
    \leq
      \sup_{x\in \Hcal}
        \bigl\lvert p(x + h) - p(x) \bigr\rvert
    \leq M \| h \| < \varepsilon
      \,.
\end{equation*}
Therefore $B(x, \delta) \subseteq \bigl\{ p\in B(p(x), \varepsilon)\bigr\}$ over
all $x\in \Hcal$ for a certain $\delta_\varepsilon > 0$, that depends on $C$ and
$\varepsilon$, but {\bf not on} $x$. Hence $p$ is uniformly continuous on $\Hcal$.

As a reminder, let $U$ be open in $\real_+ = [0, +\infty)$. For any $x\in \{p\in U\}$
there is $\varepsilon_x > 0$, such that $B(p(x), \varepsilon_x) \subseteq U$. In
turn there exists $\delta_x = \delta_{\varepsilon_x} > 0$ with
\begin{equation*}
  x \in B\bigl(x, \delta_x\bigr)
    % continuity at $x$
    \subseteq \bigl\{ p\in B(p(x), \varepsilon_x)\bigr\}
    % preimage is monotone
    \subseteq \{ p \in U \}
    \,.
\end{equation*}
% Assume the norm topology $\Tcal = \Tcal^{\|\cdot\|}_\Hcal$ on $\Hcal$.
Therefore $\{ p \in U \} = \bigcup_{x\in \{ p \in U \}} B(x, \delta_x)$, which
establishes the continuity of $p$.

% paragraph continuity (end)

\paragraph{Various preimages} % (fold)
\label{par:various_preimages}

Let's study how values of $p$ are related to the underlying set $C$, when the set
has {\bf nice} properties: $C$ is convex and has $0\in \interior{C}$, i.e. when
$p(\Hcal) \subseteq \real$, $p$ becomes continuous, and $\alpha \leq \beta$ implies
$\alpha C \subseteq \beta C$. % See \ref{pr01}, \ref{pr03}, and \ref{pr05} in summary.

% paragraph various_preimages (end)

\paragraph{Preimage of $[0, 1]$} % (fold)
\label{par:preimage_of_0_1_closed}

Suppose $C$ is such, that $p$ is continuous. Thus the set $\{p\leq 1\}$ is closed in
$\Hcal$, whence we must have $[C] \subseteq \{p\leq 1\}$, since $C \subseteq \{p\leq 1\}$.

Suppose $x\in \{p\leq 1\}$. By the assumptions on $C$ we have $x = 0 \in C \subseteq [C]$,
so suppose $x\neq 0$. For any $\lambda > 1$ we have $x\in \lambda C$, implying that
$t x \in C$ for all $t \in (0, 1)$. If for an arbitrary open ball $B(x, \delta)$ we
can find $t < 1$, such that $t x \in B(x, \delta)$, then $B(x, \delta) \cap C \neq
\emptyset$, and hence $x \in [C]$. Indeed, for any $\delta > 0$
\begin{equation*}
  t \in\Bigl(
    \max\bigl\{1 - \tfrac\delta{\|x\|}, 0\bigr\}, 1
  \Bigr)
    \Rightarrow
      \lvert t - 1\rvert < \tfrac\delta{\|x\|}
      \, \& \, t \in (0, 1)
    \Rightarrow
      t x \in B(x, \delta) \cap C
      \,.
\end{equation*}
Therefore, $[C] = \{p \leq 1\}$.
% Positive homogeneity implies that $\alpha [C] $

% paragraph preimage_of_0_1_closed (end)

\paragraph{Preimage of $[0, 1)$} % (fold)
\label{par:preimage_of_0_1_open}

Suppose $C$ is such, that $p$ is continuous. Since $\{p < 1\}$ is open by continuity
of $p$ and $\{p < 1\} \subseteq C$, we must have $\{p < 1\} \subseteq \interior{C}$.

Conversely, suppose $x \notin \{p < 1\}$. Then $x\in \{p \geq 1\}$ and in particular
$x\neq 0$. Hence for any $\lambda < 1$ we have $x \notin \lambda C$, or, equivalently,
$t x \notin C$ for all $t > 1$. If we could show that $B(x, \delta) \not\subseteq C$
for any $\delta > 0$, then $x \notin \interior{C}$. Let's make sure that there is
$t$ with $t x \in B(x, \delta)$ and $tx \notin C$:
\begin{equation*}
  t \in\Bigl(1, 1 + \tfrac\delta{\|x\|} \Bigr)
    \Rightarrow
      \lvert t - 1\rvert < \tfrac\delta{\|x\|}
      \, \& \, t \in (1, +\infty)
    \Rightarrow
      t x \in B(x, \delta) \cap \Hcal \setminus C
      % \, \& \, t x \notin C
      \,.
\end{equation*}
Therefore, $x \notin \interior{C}$, and $\interior{C} \subseteq \{p < 1\}$.

% paragraph preimage_of_0_1_open (end)

\paragraph{Preimage of $\{1\}$} % (fold)
\label{par:preimage_of_1}

We claim that for a {\bf nice} $C$, $\partial C = \{p = 1\}$. Indeed, the result
follows immediately from
\begin{equation}
  \{p = 1\}
    = \{p \leq 1\} \setminus \{p < 1\}
    = [C] \setminus \interior{C}
    = \partial C
    \,.
\end{equation}
Positive homogeneity implies, that the preimages of $[0, \alpha)$ and $[0, \alpha]$
for any $\alpha > 0$ are given by $\alpha$-inflations of the interior and the closure
of $C$, respectively. Indeed, preimages of such intervals are completely defined by
$\{p < 1\}$ and $\{p \leq 1\}$:
\begin{align*}
  \{p \leq \alpha \}
    &= \alpha \{p \leq 1\}
    = \alpha [C]
    \,, \\
  \{p < \alpha \}
    &= \alpha \{p < 1\}
    = \alpha \interior{C}
    \,, \\
  \{p = \alpha \}
    &= \alpha \{p = 1\}
    = \alpha \partial C
    \,.
\end{align*}

% paragraph preimage_of_1 (end)

\paragraph{Bounding linear functionals} % (fold)
\label{par:bounding_linear_functionals}

Let $l$ be a linear functional and $p$ be a positively homogeneous function taking
values in $[0, \infty]$. Suppose $l \leq p$ on $\Hcal$. Then for any $x\in \Hcal$ we
have $l(x) \leq p(x)$, whence $- p(x) \leq - l(x) = l(- x)$ by linearity of $l$. At
the same time $- x \in \Hcal$, implying $l(- x) \leq p(- x)$. If $x=0$, then trivially
$l(0) = 0 \leq 0 \cdot p(0)$. However, for $x\neq 0$ we get the following bound
\begin{multline*}
  - p(- x) \leq l(x) \leq p(x)
  \Rightarrow
    \\
    \lvert l(x) \rvert
    % \leq (p(x) \vee p(- x))
      \leq \| x \| \max\Bigl\{
          p\Bigl(\tfrac{x}{\|x\|}\Bigr),
          p\Bigl(- \tfrac{x}{\|x\|}\Bigr)
        \Bigr\}
      \leq \|x\| \, \sup\{p(u)\colon \|u\|\leq 1\}
    \,.
\end{multline*}
% Note that there has been no need for $p(\Hcal)\subseteq \real$ in deriving this bound.
If $p$ is the Minkowski functional of a set $C$ with $0\in \interior{C}$, then $p$
is bounded on $B[0, 1]$ and positively homogeneous. Therefore this upper bound implies
that $l$ is a bounded linear functional.

% paragraph bounding_linear_functionals (end)

\paragraph{Discussion} % (fold)
\label{par:discussion}

Obviously, if the set $\{ \lambda > 0 \colon x \in \lambda C \}$ is non-empty for
any $x \in \Hcal$, then $p(x)$ is finite-valued, because every $x$ is covered by
at least one $\lambda$-inflation.

This condition implies that $0 \in C$. Indeed, under it there must be $\lambda > 0$
such that $0 \in \lambda C$, which in turn means that $\exists z\in C$ such that
$0 = \lambda z$. Since $\lambda > 0$, we get $0 = z\in C$.

So $0 \in C$ seems necessary for finiteness. However, it is not sufficient. Consider
a cone $C = \{t z\colon z\in A\} = \bigcup_{t\geq 0} t A$ for $A = \{z\colon \|z - a\|
\leq \|a\|\}$ for a given $a \neq 0$. Since $t A = \{z\colon \|z - t a\| \leq t \|a\| \}$
for any $t\geq 0$, we still have $-a \notin C$: if it were otherwise, then $\tfrac{- a}t
\in A$ for some $t > 0$ with $1 + \tfrac1t \leq 1$ --- a contradiction.

Note that balancedness alone does not imply that $0 \in \interior{C}$, as exemplified
by taking a union of a closed convex cone and its mirrored counterpart.

% paragraph discussion (end)

\paragraph{Summary} % (fold)
\label{par:summary}

List of properties
\begin{enumerate}
  \item \label{pr01}
    $C$ is nonempty
  \item \label{pr02}
    $C$ is closed
  \item \label{pr03}
    $C$ is convex
  \item \label{pr04}
    $0 \in C$
  \item \label{pr05}
    $0$ is in the topological interior of $C$
  \item \label{pr06}
    $C$ is balanced: $\lambda C \subset C$ for all $\lvert \lambda \rvert \leq 1$
  \item \label{pr07}
    $C$ is absorbing at $0$: $\{\lambda>0\colon x \in \lambda C \} \neq \emptyset$
    for all $x\in \Hcal$
  \item \label{pr08}
    $\alpha C \subseteq \beta C$ whenever $\alpha \leq \beta$
  \item \label{pr09}
    $p(x) < +\infty$
  \item \label{pr10}
    $p(tx) = t p(x)$ for all $t > 0$
  \item \label{pr11}
    $p(tx) = \lvert t \rvert p(x)$ for all $t \in \real$
  \item \label{pr12}
    $t p(x) \leq p(t x)$ for all $t \in \real$
  \item \label{pr13}
    $p(x_1 + x_2) \leq p(x_1) + p(x_2)$ for any $x_1, x_2 \in \Hcal$
  \item \label{pr14}
    $p(0) = 0$
  \item \label{pr15}
    $A \subseteq B$ implies $\alpha A \subseteq \alpha B$
  \item \label{pr16}
    $\alpha x \in \lambda C$ iff $\lvert \alpha \rvert x \in \lambda C$ for any
    $\alpha \neq 0$ and $\lambda > 0$
  \item \label{pr17}
    $h \in C$ if and only if $-h \in C$
  \item \label{pr18}
    $p\colon \Hcal \to [0, +\infty)$ is a continuous map
  \item \label{pr19}
    $\sup\{p(u)\colon \|u\|\leq 1\} < +\infty$
\end{enumerate}
All implications tacitly assume $\{\ref{pr01}\}$:
$\{\ref{pr01}\} \Rightarrow \ref{pr10}$,
$\{\ref{pr03}, \ref{pr04}\} \Rightarrow \{\ref{pr08}, \ref{pr12}\}$,
$\{\ref{pr04}\} \Rightarrow \ref{pr14}$,
$\{\ref{pr14}\} \Rightarrow \ref{pr04}$,
$\{\ref{pr06}\} \Rightarrow \{\ref{pr11}, \ref{pr16}\}$,
$\{\ref{pr03}, \ref{pr17}\} \Rightarrow \{\ref{pr04}, \ref{pr06}\}$,
% Shown directly: if $x\in \lambda C$, then $x = \lambda z$ for some $z\in C$.
% if $\lambda > 0$, then $h = z\in C$, if $\lambda = 0$, then $\{\ref{pr17}\}$
% implies $0\in C$. Otherwise, $x = -\lvert \lambda\rvert z$ and $h = -z\in C$
% since $z\in C$. Therefore $x = (1 - \lvert \lambda\rvert) 0 + \lvert \lambda\rvert h$
% for $0, h\in C$, whence $x\in C$.
% for any $h\in C$ we have $0 = \tfrac12 (-h) + \tfrac12 h$.
$\{\ref{pr05}\} \Rightarrow \{\ref{pr04}, \ref{pr07}, \ref{pr09}\}$,
$\{\ref{pr07}\} \Rightarrow \{\ref{pr04}, \ref{pr09}\}$,
$\emptyset \Rightarrow \ref{pr15}$,
$\{\ref{pr11}\} \Rightarrow \{\ref{pr10}, \ref{pr12}\}$,
$\{\ref{pr03}\} \Rightarrow \ref{pr13}$,
$\{\ref{pr05}\} \Rightarrow \ref{pr19}$,
$\{\ref{pr03}, \ref{pr19}\} \Rightarrow \ref{pr18}$.

% paragraph summary (end)


\end{document}
