\documentclass[a4paper]{article}

\usepackage[T1]{fontenc}
\usepackage[utf8]{inputenc}

\usepackage{amsmath}
\usepackage{amssymb}
\usepackage{xcolor}
\usepackage{amsthm}
\usepackage[mathcal]{euscript}

\usepackage{url}

\newcommand{\Hcal}{\mathcal{H}}
\newcommand{\real}{\mathbb{R}}

\title{Some notes on Sokal (2010)}
\author{Nazarov Ivan}

\date{\today}

\begin{document}
\maketitle

\section*{Uniform Boundedness Principle} % (fold)
\label{sec:uniform_boundedness_principle}

In this section we treat the Hilbert space $\Hcal$ with the natural inner product
norm as a Banach space $(\Hcal, \|\cdot\|)$. Consider a collection $(l_n)_{n\geq1}$
of bounded linear functionals $\Hcal\to\real$ that is bounded at every point:
$\sup_{n\geq1} \lvert l_n(x)\rvert < +\infty$ for any $x \in \Hcal$. The following
section follows the proof given in Sokal (2010).\footnotemark
\footnotetext{arXiv:1005.1585 -- \url{https://arxiv.org/abs/1005.1585v2}}

\paragraph{Lemma} % (fold)
\label{par:lemma}

For any $x\in \Hcal$ and $\delta > 0$ we have $\delta \|l\| \leq \sup_{z\in B(x, \delta)}
\lvert l(z) \rvert$, where $B(x, \delta) = \{z\in \Hcal\colon \|z-x\| \leq \delta\}$.
Indeed, for any $\|h\| \leq 1$ we have 
\begin{equation*}
  \lvert l(h) \rvert
    = \tfrac1\delta \lvert l(\delta h) \rvert
    \leq \tfrac1{2 \delta} \bigl(
      \lvert l(x + \delta h) \rvert + \lvert l(x - \delta h) \rvert
    \bigr)
    \leq \tfrac1{2 \delta}
      2 \sup\{\lvert l(z) \rvert \colon z\in B(x, \delta)\}
      \,,
\end{equation*}
implying that the upper bound on rhs is not less than the supremum of the lhs.

% paragraph lemma (end)

\paragraph{Proof} % (fold)
\label{par:proof}

We proceed by contradiction and suppose that $\sup_{n\geq1} \|l_n\| = \infty$.

Take a sequence $L_k \uparrow +\infty$, and a pair of values $\eta, q \in (0, 1)$,
which will be specified and discussed along the proof. There exists $(n_k)_{k\geq1}
\uparrow$ such that $\|l_{n_k}\| > L_k$. Let $x_0 = 0 \in \Hcal$ and $\delta_k = q^k$.
For any $k\geq 1$ the norm $\|l_{n_k} \|$ is finite, therefore the lemma implies
\begin{equation*}
  \delta_k \eta \|l_{n_k}\|
    < \delta_k \|l_{n_k}\|
    \leq \sup\{\lvert l(z) \rvert \colon z\in B(x_{k-1}, \delta_k)\}
      \,,
\end{equation*}
which further implies the existence of $x_k \in B(x_{k-1}, \delta_k)$ with $\delta_k
\eta \|l_{n_k}\| < \lvert l_{n_k}(x_k) \rvert$. Thus, we define a sequence $(x_k)_{k\geq1}
\in \Hcal$ with $\|x_k - x_{k-1}\| \leq \delta_k$ and the just mentioned bound on
$\|l_{n_k}\|$. The constructed sequence is Cauchy: for any $p \geq k$
\begin{equation*}
  \|x_p - x_k \|
    \leq \sum_{i=k+1}^p \|x_i - x_{i-1} \|
    % \leq \sum_{i=1}^{p-k} \delta_{i+k}
    \leq \sum_{i=1}^{p-k} q^{i+k}
    % = q^{k+1} \tfrac{1 - q^{p-k}}{1-q}
    < \frac{q \delta_k}{1-q}
    \,.
    % 1 - q^n = 1 + q^1 + .. + q^{n-1} - (q^1 + q^2 + .. + q^n)
    %  = (1 - q)(1 + q^1 + .. + q^{n-1})
\end{equation*}
Since $\Hcal$ is complete, $\exists x_*\in \Hcal$ such that $\|x_k - x_*\| \to 0$.
For any $p \geq k$ we get
\begin{equation*}
  \|x_k - x_* \|
    \leq \| x_p - x_*\| + \sum_{i=k+1}^p \|x_i - x_{i-1} \|
    \leq \| x_p - x_*\| + \frac{q \delta_k}{1-q}
    \,,
\end{equation*}
and conclude that $\|x_k - x_* \| \leq \frac{q \delta_k}{1-q}$ for any $k\geq 1$.
Observe that for all $k\geq 1$
\begin{align*}
  \delta_k \eta \|l_{n_k} \|
    &< \lvert l_{n_k}(x_k) \rvert
      \leq \lvert l_{n_k}(x_*) \rvert
        + \lvert l_{n_k}(x_k - x_*) \rvert
    \\
    &\leq \lvert l_{n_k}(x_*) \rvert
      + \|x_k - x_*\| \|l_{n_k}\|
    \\
    &\leq \lvert l_{n_k}(x_*) \rvert
      + \tfrac{q}{1-q} \delta_k \|l_{n_k}\|
      \,,
\end{align*}
whence $\bigl(\eta - \tfrac{q}{1-q} \bigr) \delta_k \|l_{n_k}\| \leq \lvert l_{n_k}(x_*) \rvert$
for all $k\geq 1$.

Here we shall make the necessary specifications, that, despite affecting the particular
values, do not alter the key properties of the constructed sequence. We assume that
$1 > \eta > \tfrac{q}{1-q}$ (this implies $q < \tfrac\eta{1 + \eta}$). Since $0 \leq
L_k < \|l_{n_k}\|$ we finally get the following lower bound:
\begin{equation*}
  \tilde{L}_k
    = \bigl(\eta - \tfrac{q}{1-q} \bigr) q^k L_k
    \leq \lvert l_{n_k}(x_*) \rvert
    \,.
\end{equation*}
We are also free to require that the original $L_k \uparrow \infty$ be such that
$\tilde{L}_k \uparrow \infty$. For such choice we get the following statement: there
is $(n_k)_{k\geq1} \uparrow$ and $x_* \in \Hcal$ such that $\sup_{k\geq 1} \lvert
l_{n_k}(x_*) \rvert = +\infty$. This contradicts the original assumption that
$\sup_{n\geq 1}\lvert l_n(x) \rvert$ is bounded for any given $x\in \Hcal$.

So is it possible to meet all the requirements, outlined in the specifications above?
It is, if we pick any $\alpha, \beta > 1$ and set
\begin{itemize}
  \item $q = \tfrac1{2 (1 + \alpha)}$ and $\eta = \tfrac{\alpha q}{1-q}$
  \item $L_k = \bigl(\eta - \tfrac{q}{1-q}\bigr)^{-1} q^{-k} \beta^k$
\end{itemize}
then $\eta < 1$, $\eta - \tfrac{q}{1-q} = \tfrac{\alpha - 1}{1 + 2\alpha} > 0$,
$L_k = \tfrac{1 + 2\alpha}{\alpha - 1} (2 (1 + \alpha) \beta)^k \to \infty$,
$\delta_k = q^k \to 0$, and $\tilde{L}_k = \beta^k \to \infty$.

% paragraph proof (end)

\paragraph{Other proofs} % (fold)
\label{par:other_proofs}

There is another proof of this principle, using explicitly the continuity of the
linear functionals and the Baire Category theorem, to get a point that violates
the orignal assumptions.

% paragraph other_proofs (end)

\paragraph{Notes} % (fold)
\label{par:notes}

The uniform boundedness principle holds for arbitrarily large collection of linear
functionals and can be extended to boundned linear operators.

% paragraph notes (end)

% section* uniform_boundedness_principle (end)

\end{document}
