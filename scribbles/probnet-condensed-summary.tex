\documentclass[a4paper]{article}

\usepackage[T1]{fontenc}
\usepackage[utf8]{inputenc}

\usepackage{amsmath}
\usepackage{amssymb}
\usepackage{amsthm}
\usepackage[mathcal]{euscript}

\usepackage{url}

\newcommand{\Fcal}{\mathcal{F}}
\newcommand{\Lcal}{\mathcal{L}}
\newcommand{\Tcal}{\mathcal{T}}
\newcommand{\Bcal}{\mathcal{B}}

\newcommand{\real}{\mathbb{R}}
\newcommand{\cplx}{\mathbb{C}}

\title{Condensed notes for \url{probability.net}}
\author{Nazarov Ivan}

\date{\today}

\begin{document}

\maketitle


\paragraph{tut 11} % (fold)
\label{par:tut_11}

Complex measures form a normed $\cplx$-vector space $M_1(\Omega, \Fcal)$ and their
total variation is always a finite measure;
%
$L^1_\cplx(\mu) \equiv L^1_\cplx(|\mu|)$;
%
the norm is $\|\mu\| = |\mu|(\Omega)$;
%
real valued bounded linear functionals on complex measures give signed measures;
%
their real~/~imaginary total variation can be used to decompose the complex measure
into a $\cplx$-linear combination;
%
reverse triangle inequality: there is $S \subseteq \{1,\, \ldots,\,N\}$ such that
$\sum_{i=1}^N |z_i| \leq \pi |\sum_{i \in S} z_i|$.

% paragraph tut_11 (end)


\paragraph{tut 12} % (fold)
\label{par:tut_12}

Absolute continuity of measures has a $\varepsilon - \delta$ continuity set-theoretic
equivalent;
%
if for a finite measure $\mu$ and $f \in L^1_\cplx(\Omega, \Fcal, \mu)$, we have
$\tfrac1{d\mu(E)} \int_E f d\mu \in S$ for $S$ closed in $\cplx$, then $f \in S$
$\mu$-as;
%
Radon-Nikodym derivative $h$ of a complex~/~signed~/~finite measure $\nu$ on
$(\Omega, \Fcal)$ wrt. $\sigma$-finite measure $\mu$ is an element of $L^1_\cplx(\mu)$
and respects the co-domain of $\nu$;
%
the same holds for a pair of $\sigma$-finite measures, but $h\colon \Omega \to [0, +\infty]$;
%
derivative of a complex measure wrt. its total variation exists and its modulus~/~absolute
value is $1$ almost surely;
%
set convergence holds for complex measures (via DCT);
%
for any measure $\mu$ if $d\nu = f d\mu$ for some complex $\mu$-integrable $f$, then
$\nu$ is a complex measure, $d|\nu| = |f| d\mu$ and for any bounded measurable $g$ or
$g \in L^1_\cplx(\nu)$ we have $\int g d\nu = \int g f d\mu$;
%
Complex Lebesgue integral is defined as $\int f d\mu = \int f h d|\mu|$,
$d\mu = h d|\mu|$;
%
Complex Lebesgue is bilinear wrt. function and measure arguments;
%
ordinary Lebesgue is also monotone wrt. measures and linear wrt. nonnegative finite
linear combinations of measures;
%
Complex integral can be decomposed into a linear combination of ordinary Lebesgue
integrals wrt. finite measures based on total-variation (unique for any given complex
measure);
%
Basically all commons sense operations and manipulations with complex integrands
and complex measures are legal.

% paragraph tut_12 (end)


\paragraph{tut 13} % (fold)
\label{par:tut_13}

Simple $\cplx$-valued functions that are also in $L^p_\cplx(\Omega, \Fcal, \mu)$ are
dense in $L^p_\cplx(\mu)$;
%
for any finite measure $\mu$ on metrizable space $(\Omega, \Tcal)$
(metrizability -- to get open neighborhoods of closed sets) for any measurable
$E \in \Bcal(\Omega)$ and $\varepsilon > 0$ there are $F$ -- closed and $U$ -- open,
$F \subseteq E \subseteq U$ with $\mu(U \setminus F) \leq \varepsilon$;
%
bounded continuous functions $C_\cplx^b(\Omega, \Tcal)$ on metrizable spaces
(or T4 for Urhyson lemma) approximate measurable indicator functions;
%
bounded continuous functions on metrizable spaces (T4 not enough due to measure!)
are dense in $L^p_\cplx(\Omega, \Bcal(\Omega), \mu)$ for $\mu$ finite measure and
$p < +\infty$;
%
Portmanteau lemma: complex measures on metrizable space coincide if their integrals
are equal for all continuous bounded real-valued functions;
%
locally finite borel measure: every point has an open neighborhood with finite measure;
%
compact sets have finite measure for any locally finite measure;
%
metrizable compact ($\sigma$-compact) topological space is separable;
%
in a metrizable topological space separability is equivalent to a countable base;
%
in a topological space with a locally finite borel measure a subset of a base with finite
measure is still a topological base;
%
on a metrizable $\sigma$-compact topological space any locally finite measure is
inner and outer regular;
%
strongly $\sigma$-compact topological space is the same as $\sigma$-compact and
locally compact;
%
in a metrizable locally compact space for any $K$ -- compact, $U$ -- open, $K \subseteq U$,
there is a continuous function $\phi$ with compact support ($[\{\phi \neq 0\}]$
-- compact in $\Omega$) such that $1_K \leq \phi \leq 1_U$;
%
on a metrizable strongly $\sigma$-compact topological space continuous functions
with compact support are dense in $L^p_\cplx(\Omega, \Bcal(\Omega), \mu)$.

Locally compact -- every open neighborhood has a compact closure;
%
$\sigma$-compact -- there exists an increasing family of compact sets that covers $\Omega$;
%
strongly $\sigma$-compact -- there is an increasing sequence of open sets with compact
closure that covers $\Omega$;

for $
  H\vert_X
    = \{A\cap X\colon A \in H\}
$ we have $
  T_X
    = T(H) \vert_X
    = T(H \vert_X)
$;
%
$H$ is a topological base on $\Omega$ if $H$ cover $\Omega$ and $x \in U \cap V$ implies $
  \exists W \in H
  \,\colon\, x \in W \subseteq U \cap V
$;
%
$
\mathcal{T}(H)
  = \{
    \cup_{V \in \Gamma} V
    \colon \Gamma \subseteq H
  \}
  = \{
    U \subseteq \Omega
    \colon \forall x\in U\, \exists W\in H
    \colon x \in W \subseteq U
  \}
$.

% paragraph tut_13 (end)


\paragraph{tut 14} % (fold)
\label{par:tut_14}

Total variation (tv) of a $b\colon \real^+ \to \cplx$ map is non-decreasing, preserves
right~/~left continuity if any, and
$|b|(t) - |b|(s) = \sup_{t_i \in [s, t]} \sum_i |b(t_i) - b(t_{i-1})|$;
%
tv of a nonnegative non-decreasing map is the map itself;
%
tv of a map is sub-additive and preserves order;
%
nondecreasing nonnegative $\real$-valued maps (of finite variation) make Stieltjes
measures, complex maps of bounded variation --- complex Stieltjes measures,
$db((s,t]) = b(t) - b(s)$, $db(\{0\}) = b(0)$;
%
tv of a complex Stieltjes measure is the Stieltjes measure of the tv, i.e. $|db| = d|b|$;
%
total variations can be used to linearly decompose maps of finite variation;
%
for measures on $\real^+$ to be ordered it is sufficient for them to be ordered on
half-intervals and at zero;
%
tv of a right~/~left continuous map can be computed on partitions
$t_k = \tfrac{k t}{2^n}$, $k=0,\, \ldots,\,2^n$;
%
cadlag complex maps are bounded on bounded sets: for some $M\geq 0$
$|b(t-) \vee b(t)| \leq M$ for all $t \in [0, T]$.

% paragraph tut_14 (end)


\paragraph{tut 15} % (fold)
\label{par:tut_15}

Stieltjes (local) $L^1$-space $L^1_\cplx(b)$ ($L^{1,loc}_\cplx(b)$) for right continuous
({\bf rc}) $b\colon \real^+ \to \cplx$ of finite variation is
$L_\cplx^1(\real^+, \Bcal(\real^+), d|b|)$ ($L_\cplx^1(d|b|^{[0,t]})$);
%
for an {\bf rc} nonnegative nondecreasing $a$ and $f \in L^1_\cplx(a)$ (or $f \geq 0$ in
$L^{1,loc}_\real(a)$) we have $f.a(t) = \int_0^t f da$ {\bf rc} map of bounded (finite)
variation and $d(f.a) = f da$;
%
for {\bf rc} nonnegative nondecreasing $a$ and $f \in L^1_\cplx(a)$ we have
$|f.a| = |f|.a$ with $\Delta f.a = f\Delta a$, $\Delta a = a(t) - a(t-)$;
%
for $a\colon \real^+ \to \real^+$ {\bf rc} nonnegative nondecreasing and
$b\colon \real^+ \to \cplx$ {\bf rc} of finite variation there is $f \in L^1_\cplx(a)$
$da$-as unique such that $b = f.a$ and $f$ preserves the co-domain of $b$;
%
for $b\colon \real^+ \to \cplx$ {\bf rc} of finite variation there exists
$h \in L^{1,loc}_\cplx(b)$ with $|h|=1$ and $b = h.|b|$ and for $f \in L^1_\cplx(b)$
the Stieltjes integral of $f$ is $f.b = (f h).|b|$, $|f.b| = |f|.|b|$ and
$\Delta f.b = f \Delta b$;
%
for $g \in L^{1,loc}_\cplx(b)$ we have $f.(g.b) = (fg).b$ for any
$f \in L^{1,loc}_\cplx(g.b) \Leftrightarrow fg \in L^{1,loc}_\cplx(b)$ or bounded;
%
$f.b$ is bilinear for $f \in L^{1,loc}_\cplx(b)$ and $b\colon \real^+ \to \cplx$
is {\bf rc} of finite variation;
%
$f.b = f.|b_1|^+ - f.|b_1|^- + i(f.|b_2|^+ - f.|b_2|^-)$ for $b_1 = \Re(b)$, $b_2 = \Im(b)$;
%
for $a\colon \real^+ \to \real^+$ {\bf rc} nonnegative and non decreasing the inverse
$c_a(t) = \inf\{s \geq 0 \mid t < a(s) \}$ is nonnegative, nondecreasing and right
continuous, sort of generalized inverse, and $f.a(t) = ((f \circ c_a).s)(a(t))$ for
$f \in L^{1,loc}_\cplx(a)$ and also $a = c_{c_a}$;
%
nondecreasing maps are always measurable.

% paragraph tut_15 (end)


\paragraph{tut 16} % (fold)
\label{par:tut_16}

Recall:
$\limsup_{x\to x_0} = \inf_{U\ni x_0} \sup_{x\in U}$,
$A_n~\text{i.o.} = \cap_{n\geq 0} \cup_{k\geq n} A_k$,
$A_n~\text{ev.} = \cup_{n\geq 0} \cap_{k\geq n} A_k$;
%
lsc functions ($\{\lambda < f\}$ -- open in $\Omega$) are closed under addition,
scaling by nonnegative values, and arbitrary suprema;
%
lsc = - usc;
%
$1_U$, $U$ open, is lsc;
%
$1_F$, $F$ -- closed, is usc;
%
all usc~/~lsc maps are measurable wrt. Borel $\sigma$-algebra;
%
on a metrizable $\sigma$-compact space with a locally finite measure $\mu$ for any
$f \in L^1_\real(\Omega, \Fcal, \mu)$ and any $\varepsilon > 0$ there are lsc $v$ and
usc $u$ ($\mu$-as having counterparts in $L^1_\real(\mu)$) such that $u \leq f \leq v$
and $\int (v - u) d\mu \leq \varepsilon$ (Viteli-Caratheodory);
%
a connected topological space has $\emptyset$ and $\Omega$ as the {\bf only} closed
and open;
%
a space is connected iff whenever $\Omega = A \uplus B$ then either $A$ or $B$ is
$\emptyset$;
%
an interval is a subset $G$ of $\real$ such that for all $a \leq b \in G$ we have
$[a, b] \subseteq G$;
%
$G$ is an intervals iff $G$ is either $(a, b]$, $(a, b)$, $[a, b)$, or $[a, b]$;
%
a subset of $\real$ is an interval iff it is connected;
%
images of connected subsets via a continuous map are connected;
%
if $\Omega$ is connected and $f$ continuous, then $[a, b] \subseteq f(\Omega)$ implies
that for any $c \in [a, b]$ there is $z \in \Omega$ with $c = f(z)$;
%
if $a < b$, and $f$ is differentiable everywhere at $[a, b]$ with
$\int_a^b | f'(x)| dx < + \infty$, then $f(b) - f(a) = \int_a^b f'(s) ds$;
%
on $(\real^n, \Bcal(\real^n))$ for any $\alpha > 0$ we have
$dx({k_\alpha \in B})
  = \alpha^{-n} dx(B)
  = \lvert J_{k_\alpha} \rvert^{-1} dx(B)$;
%
for complex measure $\mu$ on $(\real^n, \Bcal(\real^n))$ the maximal function
$M\mu(x) = \sup_{\varepsilon>0}
  \tfrac{|\mu|(B(x,\varepsilon))}
        {dx(B(x,\varepsilon))}$
is nonnegative and lsc;
%
(constructive) for a collection $(B(x_i,\varepsilon_i))_{i=1}^n$, $\varepsilon_i > 0$
there exists an $S \subseteq \{1,\,\ldots,\,n\}$ such that
$\cup_{i=1}^n B(x_i,\varepsilon_i))_{i=1}^n
  \subseteq \cup_{i\in S} B(x_i, 3\varepsilon_i)$
and $(B(x_i,\varepsilon_i))_{i\in S}$ are pairwise disjoint;
%
for a complex measure $\mu$ and any $\lambda > 0$ we have
$dx(\{\lambda < M\mu\}) \leq \tfrac{3^n}{\lambda} \|\mu\|$;
%
the maximal function of $f \in L^1_\cplx(\real^n, \Bcal(\real^n), dx)$ is
$M f = M \mu$ for $d\mu = f dx$;
%
for any $f \in L^1_\cplx(dx)$ we have
$\bigl\{
  \lim_{\varepsilon \downarrow 0}
    \tfrac{\int_{B(x, \varepsilon)} |f(x) - f(y)| dy}
          {dx(B(x, \varepsilon)))}
  = 0\bigr\}$ $dx$-a.s.

% paragraph tut_16 (end)


\paragraph{tut 17} % (fold)
\label{par:tut_17}

The space of $n\times n$ matrices $M_n(K)$, $K = \real$ or $\cplx$, is the same as
the space $M'_n(K)$ of all finite products of diagonal scaling ($H_\alpha e_1 = \alpha e_1$,
$H_\alpha e_i = e_i$, $\alpha \in K$), row swapping $P_{ij}$ ($P_{ij} e_i = e_j$,
$P_{ij} e_j = e_i$, and $P_{ij} e_l = e_l$) and pivoting ($U e_1 = e_1 + e_2$, $U e_j = e_j$)
matrices;
%
if matrices with $1$ in the top left corner is in $M'_n(K)$, then $M_n(K)\subseteq M'_n(K)$;
%
if matrices with $e_1$ in its first column is in $M'_n(K)$, then all matrices with
$1$ in the top left corner are in $M'_n(K)$;
%
the distribution of $X\colon (\Omega, \Fcal) \to (\Omega', \Fcal')$ under $\mu$ (possibly
complex), or the {\bf image measure} of $\mu$ under $X$, or the {\bf pushforward} of
$\mu$, $X_\sharp\mu$, is $X(\mu)(E) = \mu^X(E) = \mu(\{X \in E\}) = \mu(X^{-1}(E))$,
and we have $Y(X(\mu)) = Y(\mu^X)$;
%
$|\mu^X| \leq |\mu|^X$;
%
a complex measure $\mu$ on $(\real^n, \Bcal(\real^n))$ is translation invariant iff
$\tau_\alpha(\mu) = \mu$ for $\tau_\alpha(x) = \alpha + x$;
%
$\tau_\alpha(dx) = dx$, $k_\alpha(dx) = \alpha^{-n} dx$ for $x\mapsto k_\alpha(x) =
\alpha x$, $a > 0$;
%
for $X\colon (\Omega, \Fcal) \to (\Omega', \Fcal')$ we have $\int_{\Omega} (f \circ X)
d\mu = \int_{\Omega'} f dX(\mu)$ for measure $\mu$ and measurable $f\colon \Omega' \to
[0, +\infty]$ or $f\in L^1_\cplx(\Omega', \Fcal', X(\mu))$ (iff $f\circ X \in
L^1_\cplx(\Omega, \Fcal, \mu)$), or for a complex measure $\mu$ and integrable
$f$ (the STM);
%
the \textbf{s}tandard \textbf{m}achine \textbf{a}rgument: indicator $\to$ simple
(linearity) $\to$ nonnegative (mct$\uparrow$, or dct$\to$+majorant) $\to$ general
integrable, if the measurable sets are involved, then use Dynkin's lemma: if a
$\lambda$-system includes a $\pi$-system, then it includes the $\sigma$-algebra
of the latter;
%
if $\mu$ is a translation-invariant measure $(\real^n, \Bcal(\real^n))$, then there
is a unique $\alpha \geq 0$, such that $\mu = \alpha dx$ (due to translation invariance
the scaling in front of $\mu$ becomes the Lebesgue measure);
%
$T\colon \real^n \to \real^n$ is a linear bijection iff it has a non-singular matrix
representation in, e.g., canonical bases in domain and image spaces;
%
the linearity of $T$ makes the image measure $\mu = T(dx)$ translation invariant in
$\real^n$ with $\mu = \Delta(T) dx$;
%
the matrix of $T$ being composed of invertible simple matrices makes $\Delta(T) \neq
0$ and $\Delta(T_1 \circ T_2) = \Delta(T_1) \Delta(T_2)$;
%
ultimately $\Delta(T) = \lvert \det T^{-1} \rvert = \lvert \det T \rvert^{-1}$ due
to the determinant of a product being equal to the product of determinants;
%
$dx(T(B)) = \lvert \det T\rvert dx(B)$;
%
any linear subspace $V$ of $\real^n$ is closed and $dx(V) = 0$ when $\dim V \leq n-1$;

% paragraph tut_17 (end)


\paragraph{tut 18} % (fold)
\label{par:tut_18}

The norm topology of the $K$-normed vector space is the metric topology induced by
the norm metric;
%
if $E$ and $F$ are $K$-normed spaces and $l\colon E \to F$ is a linear map, then
$l$ is continuous iff $l$ is continuous at $0$, iff there is $M\geq 0$ such that
$\|l(x)\| \leq M \|x\|$ for all $x\in E$, iff
$\|l\| = \sup\{\|l(x)\|\colon x\in E,\, \|x\|=1\} < +\infty$;
%
the norm $\|l\|$ of a linear map $l$ is $\sup\{\|l(x)\|\colon x\in E,\, \|x\| \leq 1\}$,
$\sup\{\tfrac{\|l(x)\|}{\|x\|} \colon x\in E,\, x\neq 0\}$, and
$\inf\{M\geq 0\colon \|l(x)\| \leq M \|x\|\,\,\forall{x\in E} \}$;
%
for all $x\in E$ we have $\|l(x)\| \leq \|l\| \|x\|$, and $(\Lcal_K(E, F), \|\cdot\|)$
is a $K$-normed space.

For $E, F$ be $\real$-normed spaces and an open $U\subseteq E$, the map $\phi\colon E\to F$
is differentiable at $a\in U$ iff there is $d\phi(a) \in \Lcal_\real(E, F)$ so that for any
$\varepsilon > 0$ there is $\delta > 0$ such that for all $h\in E$ with $\|h\| \leq \delta$
and $a + h \in U$ we have $\|\phi(a+h) - \phi(a) - l(h) \| \leq \varepsilon \|h\|$;
%
the linear map $d\phi(a)$ is unique, and the map $\phi$ is differentiable on $U$ if
it is differentiable at every $a\in U$;
%
$\phi\colon U \to F$ is $C^1$ if it is differentiable on $U$ and
$d\phi\colon U \to (\Lcal_\real(E, F), \|\cdot\|)$ is continuous;
%
for $\real$-normed spaces $E, F, G$ with $U$ -- open in $E$, $V$ -- open in $F$, if
maps $\phi\colon U\to F$ and $\psi\colon V\to G$ are differentiable at $a\in U$ and
$\phi(a)\in V$ with $\phi(U) \subseteq V$, then the map $\psi\circ \phi$ is differentiable
at $a$ and $d(\psi\circ\phi)(a) = d\psi(\phi(a)) \circ d\phi(a)$;
%
norms on finite product spaces are topologically equivalent;
%
if norms $\|\cdot\|$ and $N_E(\cdot)$ induce the same topology, then the
$id_E\colon (E, \|\cdot\|) \to (E, N_E(\cdot))$ is continuous, and the norms are equivalent:
there exist $m_E, M_E > 0$ such that $m_E\|\cdot\| \leq N_E(\cdot) \leq M_E \|\cdot\|$;
%
if maps are differentiable at $a$ so are their linear combinations;
%
the differential does not depend on the choice of norms in finite dimensional spaces;
%
if $\phi\colon U\to F$ and $\psi\colon V\to G$ are $C^1$, then so their composition is
$C^1$ as well;
%
if $E$ is an $\real$-normed space, $f\colon [a, b] \to E$ and $g\colon [a, b] \to \real$
are differentiable at every $(a, b)$ with $\|f'(t)\| \leq g'(t)$, then
$\|f(b) - f(a)\| \leq g(b) - g(b)$;
%
$\Rightarrow$ if $\phi\subseteq \real^n \to E$, $\real$-normed, is differentiable at
$a\in U$ then all $\tfrac{\partial \phi}{\partial x_i}(a)$ exist at $a$, furthermore
$d\phi(a)(h) = \sum_i \tfrac{\partial \phi}{\partial x_i}(a) h_i$;
%
if $d\phi$ exists on $U$ and is continuous at $a$, then
$\tfrac{\partial \phi}{\partial x_i}(\cdot)$ are continuous at $a$;
%
$\Leftarrow$ if $\phi$ has all $\tfrac{\partial \phi}{\partial x_i}(\cdot)$ on $U$ and
they are continuous at $a\in U$, then $\phi$ is differentiable at $a$ (only existence!);
%
if $\tfrac{\partial \phi}{\partial x_i}(\cdot)$ are continuous on $U$ then so is $d\phi$
on $U$;
%
finally $\phi\colon U\to E$ is $C^1$ on $U$ iff each
$a\to \tfrac{\partial \phi}{\partial x_i}(a)$ is $U$-$E$ continuous;
%
$l\in \Lcal(E, F)$ is $C^1$ and $d(l\big\vert_U)(a) = l$ for any $a\in U$ -- open in $E$;
%
$\phi\colon U \to F = \prod_{i=1}^p F_p$ is differentiable at $a$ iff each
$\phi_i\colon U\to F_i$ is, $\phi$ is $C^1$ on $U$ iff each $\phi_i$ is $C_1$ on $U$,
and $d\phi(a) = (d\phi_i(a))_{i=1}^n \in \Lcal(E, F)$;
%
if $\phi\colon U \to \real^n$ then
$d\phi(a) = (\tfrac{\phi_c}{\partial x_r})_{r,c}^{n,n} \in \real^{n\times n}$;
%
the Jacobian of $\phi$ is $J(\phi)(a) = \det d\phi(a)$ --- the determinant of the
partial derivative matrix.

For $\Omega$ and $\Omega'$ open in $\real^n$ a bijection $\phi\colon \Omega \to \Omega'$
is a $C^1$-diffeomorphism between $\Omega$ and $\Omega'$ iff $\phi\colon \Omega \to \real^n$
and $\phi^{-1}\colon \Omega' \to \real^n$ are both $C^1$;
%
if $\phi$ is a $C^1$-diffeomorphism ($\psi = \phi^{-1}$) then
$d\phi(\psi(x)) \circ d\psi(x) = id_{\real^n}$ and
$d\psi(\phi(x)) \circ d\phi(x) = id_{\real^n}$ and
$J(\phi) = \tfrac1{J(\psi) \circ \phi} \neq 0$;
%
if $d\phi(a) = id_{\real^n}$, then for all $x\in \Omega'$
$\lim_{\varepsilon\downarrow 0}
  \tfrac{\phi(dx\big\vert_{\Omega})(B(x, \varepsilon))}
        {dx\big\vert_{\Omega'}(B(x, \varepsilon))} = 1$;
%
the matrix $A = d\phi(x)$ is non-singular, and $\tilde{\phi} = \phi \circ A$ is $C^1$
diffeomorphism between $\Omega'' = A^{-1}(\Omega)$ and $\Omega'$ with
$\tilde{\psi} = A^{-1} \circ \psi$ and
$d\tilde{\phi}(x) = id_{\real^n} \in \Lcal(\real^n, \real^n)$;
%
hence
$\lim_{\varepsilon\downarrow 0}
  \tfrac{\phi(dx\big\vert_{\Omega})(B(x, \varepsilon))}
        {dx\big\vert_{\Omega'}(B(x, \varepsilon))}
  = \lvert \det J(\psi)(x) \rvert$ for all $x\in \Omega'$;
%
finally $\phi(dx\big\vert_{\Omega}) \ll dx\big\vert_{\Omega'}$ for a $C^1$-diffeomorphism
$\phi$ between $\Omega$ and $\Omega'$ and the Radon-Nikodym derivative is
$a \to \lvert \det J(\psi)(a) \rvert$;
%
the Jacobian formula is
$\int_\Omega (f\circ \phi) dx\big\vert_{\Omega}
  = \int_{\Omega'} f \, \phi(dx\big\vert_{\Omega})
  = \int_{\Omega'} f \lvert J(\phi^{-1})\rvert dx\big\vert_{\Omega'}$
and
$\int_{\Omega'} f dx\big\vert_{\Omega'}
  = \int_{\Omega} (f\circ \phi) \, \phi^{-1}(dx\big\vert_{\Omega})
  = \int_{\Omega} (f\circ \phi) \lvert J(\phi)\rvert dx\big\vert_{\Omega}$
-- true for nonnegative $f$, $f\in \Lcal_\cplx^1(\Omega', \Bcal(\Omega'), dx)$
iff $(f \circ \phi) \lvert J(\phi) \rvert \in \Lcal_\cplx^1(\Omega, \Bcal(\Omega), dx)$,
and $f \circ \phi \in \Lcal_\cplx^1(\Omega, \Bcal(\Omega), dx)$
iff $f \lvert J(\phi^{-1}) \rvert \in \Lcal_\cplx^1(\Omega', \Bcal(\Omega'), dx)$;
%
{\bf pushforward}: $\phi_\sharp dx_\Omega = \lvert \det J(\phi^{-1}) \rvert dx_{\Omega'}$;
%
the integral w.r.t. measure $\mu$ of a {\bf pullback} of $f\colon \Omega'\to \real$ by
$\phi\colon \Omega\to\Omega'$ is the integral of $f$ w.r.t the {\bf pushforward}
of $\mu$ by $\phi$: $\int_\Omega f\circ \phi d\mu = \int_{\Omega'} f d\phi_\sharp\mu$.
% paragraph tut_18 (end)


\paragraph{tut 19} % (fold)
\label{par:tut_19}



% paragraph tut_19 (end)


\paragraph{tut 20} % (fold)
\label{par:tut_20}



% paragraph tut_20 (end)
\end{document}
