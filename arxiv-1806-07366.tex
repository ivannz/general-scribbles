\documentclass[a4paper]{article}

\usepackage[T1]{fontenc}
\usepackage[utf8]{inputenc}

\usepackage{amsmath}
\usepackage{amssymb}
\usepackage{amsthm}
\usepackage[mathcal]{euscript}

\usepackage{url}

\newcommand{\real}{\mathbb{R}}

\newcommand{\T}{\mathrm{T}}
\DeclareMathOperator{\Tr}{Tr}

\title{Some notes on ArXiv:\url{1806.07366}}
\author{Nazarov Ivan}

\date{\today}

\begin{document}
\maketitle

\noindent
Consder the Cauchy problem $\dot{z} = f(t, z)$, $z(t_0) = z_0 \in \real^d$. Let $f$ be
jointly continuous on an open connected domain $D \subseteq \real \times \real^d$ and
Lipschits w.r.t. $z$ uniformly over $t$ on any closed and bounded subset of $D$. Then
there is $\delta > 0$ such that the Cauchy problem admits a solution on $[t_0 - \delta,
t_0 + \delta]$, and solutions defined on onverlapping intervals coincide over the
intersection.

If $f$ depends on a parameter is a continuous or differentiable manne uniformly on $t$,
then the solution to ODE is also continuous~/~differentiable w.r.t. that parameter. This
follows from the successive approximations used in the proof of Picard's existence and
uniqueness theorem and Lipschits condition.\footnotemark
\footnotetext{G. Petrovskiy, Lecture on Ordinary Differential Equations,
\url{http://www.rfbr.ru/rffi/ru/books/o_17815#99}}.

% The solution of this ode coincides with $\hat{z}(t) = z_0 + \xi(t - t_0)$, where $\xi$
% solves $\dot{\xi} = g(s, \xi)$, $\xi(0) = 0$, and $g(s, \xi) = f(s + t_0, \xi + z_0)$.
% This follows from the uniqueness of ode solutions, since $f$ is continuous.

Now suppose $z(t_0 + \epsilon) = T_\epsilon (z(t_0))$ for a small $\epsilon > 0$. Then
\begin{equation}
  T_\epsilon(z(t_0))
    = z_0 + \int_0^\epsilon f(t + \tau, z(t) + \xi(\tau)) d\tau
    \,.
\end{equation}
The 1st-order Taylor series expansion of $z(t_0 + \epsilon)$ w.r.t $\epsilon$ is
\begin{equation}
  T_\epsilon (z(t_0))
    = z_0 + \tfrac{d}{d\epsilon} T_\epsilon (z(t_0)) \Big\vert_{\epsilon = 0} \, \epsilon
      + \mathcal{O}(\epsilon^2)
    = z_0 + f(t_0, z_0) \, \epsilon
      + \mathcal{O}(\epsilon^2)
    \,.
\end{equation}
Differentiating the linear terms w.r.t. $z(t_0)=z_0$ results in
\begin{equation}
  \tfrac{\partial}{\partial z_0} T_\epsilon (z(t_0))
    = I + \epsilon \, \nabla_z f(t_0, z) \big \vert_{z=z_0}
    \,,
\end{equation}

Let $p$ be some density or a likelihood. Then by the Jacobian formula
\begin{equation}
  \log p(z(t_0 + \epsilon))
    = \log p(z_0)
    - \log \bigl \lvert
        \det \tfrac{\partial}{\partial z} T_\epsilon (z)
      \bigr \rvert \Big \vert_{z=z_0}
  \,.
\end{equation}
The derivative of $\log p(z(t))$ w.r.t. $t$ at $t_0$ is
\begin{equation}
  \tfrac{\partial}{\partial t}
    \log p(z(t)) \bigg\vert_{t = t_0}
    = \tfrac{\partial}{\partial \epsilon}
      \log p(z(t_0 + \epsilon)) \bigg\vert_{\epsilon = 0}
    = - \tfrac{\partial}{\partial \epsilon} \log \bigl \lvert
        \det \tfrac{\partial}{\partial z} T_\epsilon (z)
      \bigr \rvert \bigg \vert_{\substack{
        z=z_0 \\ \epsilon = 0}}
      \,.
\end{equation}
Let's differentiate the log-det (we can ignore the absolute value here since,
in a small neighborhood the matrix has to be nonsingular)
\begin{align}
  \tfrac{\partial}{\partial \epsilon}
    \log \det \tfrac{\partial}{\partial z} T_\epsilon (z)
    &= \Tr{\bigl(
        \bigl( \tfrac{\partial}{\partial z} T_\epsilon (z) \bigr)^{- \T}
        \tfrac{\partial}{\partial \epsilon}
          \tfrac{\partial}{\partial z} T_\epsilon (z)
      \bigr)} \bigg\vert_{\epsilon = 0}
    \notag \\
    &= \Tr{ I^{- \T} \nabla_z f(t_0, z) \big \vert_{z=z_0}}
    \,,
\end{align}
whence
\begin{equation}
  \tfrac{\partial}{\partial t}
    \log p(z(t)) \bigg\vert_{t = t_0}
    = - \Tr{\nabla_z f(t_0, z_0) \big \vert_{z=z_0}}
      \,.
\end{equation}


\end{document}
